\subsection{qa\+\_\+split\+\_\+fsk\+\_\+cc.\+py}
\label{qa__split__fsk__cc_8py_source}\index{/home/erik/prefix/default/src/gr-\/radar-\/dev/python/qa\+\_\+split\+\_\+fsk\+\_\+cc.\+py@{/home/erik/prefix/default/src/gr-\/radar-\/dev/python/qa\+\_\+split\+\_\+fsk\+\_\+cc.\+py}}

\begin{DoxyCode}
00001 \textcolor{comment}{#!/usr/bin/env python}
00002 \textcolor{comment}{# -*- coding: utf-8 -*-}
00003 \textcolor{comment}{# }
00004 \textcolor{comment}{# Copyright 2014 Communications Engineering Lab, KIT.}
00005 \textcolor{comment}{# }
00006 \textcolor{comment}{# This is free software; you can redistribute it and/or modify}
00007 \textcolor{comment}{# it under the terms of the GNU General Public License as published by}
00008 \textcolor{comment}{# the Free Software Foundation; either version 3, or (at your option)}
00009 \textcolor{comment}{# any later version.}
00010 \textcolor{comment}{# }
00011 \textcolor{comment}{# This software is distributed in the hope that it will be useful,}
00012 \textcolor{comment}{# but WITHOUT ANY WARRANTY; without even the implied warranty of}
00013 \textcolor{comment}{# MERCHANTABILITY or FITNESS FOR A PARTICULAR PURPOSE.  See the}
00014 \textcolor{comment}{# GNU General Public License for more details.}
00015 \textcolor{comment}{# }
00016 \textcolor{comment}{# You should have received a copy of the GNU General Public License}
00017 \textcolor{comment}{# along with this software; see the file COPYING.  If not, write to}
00018 \textcolor{comment}{# the Free Software Foundation, Inc., 51 Franklin Street,}
00019 \textcolor{comment}{# Boston, MA 02110-1301, USA.}
00020 \textcolor{comment}{# }
00021 
00022 \textcolor{keyword}{from} gnuradio \textcolor{keyword}{import} gr, gr\_unittest
00023 \textcolor{keyword}{from} gnuradio \textcolor{keyword}{import} blocks
00024 \textcolor{keyword}{import} radar\_swig \textcolor{keyword}{as} radar
00025 
00026 \textcolor{keyword}{class }qa_split_fsk_cc (gr\_unittest.TestCase):
00027 
00028     \textcolor{keyword}{def }setUp (self):
00029         self.tb = gr.top\_block ()
00030 
00031     \textcolor{keyword}{def }tearDown (self):
00032         self.tb = \textcolor{keywordtype}{None}
00033 
00034     \textcolor{keyword}{def }test_001_t (self):
00035         \textcolor{comment}{# set up fg}
00036         test\_len = 1200
00037         
00038         samp\_rate = 4000
00039         samp\_per\_freq = 2
00040         blocks\_per\_tag = 100
00041         freq\_low = 0
00042         freq\_high = 200
00043         amplitude = 1
00044         
00045         samp\_discard = 1
00046         
00047         src = radar.signal\_generator\_fsk\_c(samp\_rate, samp\_per\_freq, blocks\_per\_tag, freq\_low, freq\_high, 
      amplitude)
00048         head = blocks.head(8,test\_len)
00049         split = radar.split\_fsk\_cc(samp\_per\_freq,samp\_discard)
00050         snk0 = blocks.vector\_sink\_c()
00051         snk1 = blocks.vector\_sink\_c()
00052         snk = blocks.vector\_sink\_c()
00053         
00054         self.tb.connect(src,head,split)
00055         self.tb.connect((split,0),snk0)
00056         self.tb.connect((split,1),snk1)
00057         self.tb.connect(head,snk)
00058         self.tb.run ()
00059         
00060         \textcolor{comment}{# check correct length of data}
00061         nblock = test\_len/2/samp\_per\_freq
00062         self.assertEqual(len(snk0.data()),nblock*(samp\_per\_freq-samp\_discard))
00063         self.assertEqual(len(snk1.data()),nblock*(samp\_per\_freq-samp\_discard))
00064         
00065         \textcolor{comment}{# split data self}
00066         data = snk.data()
00067         data0 = []
00068         data1 = []
00069         k=0
00070         \textcolor{keywordflow}{while} k<len(data):
00071             k = k+samp\_discard
00072             data0.append(data[k])
00073             k = k+1
00074             k = k+samp\_discard
00075             data1.append(data[k])
00076             k = k+1
00077         
00078         \textcolor{comment}{# check if data0/1 is same as split from block}
00079         self.assertComplexTuplesAlmostEqual(data0,snk0.data(),10)
00080         self.assertComplexTuplesAlmostEqual(data1,snk1.data(),10)
00081             
00082         
00083 
00084 \textcolor{keywordflow}{if} \_\_name\_\_ == \textcolor{stringliteral}{'\_\_main\_\_'}:
00085     gr\_unittest.run(qa\_split\_fsk\_cc, \textcolor{stringliteral}{"qa\_split\_fsk\_cc.xml"})
\end{DoxyCode}
