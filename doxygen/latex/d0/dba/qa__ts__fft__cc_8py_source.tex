\subsection{qa\+\_\+ts\+\_\+fft\+\_\+cc.\+py}
\label{qa__ts__fft__cc_8py_source}\index{/home/erik/prefix/default/src/gr-\/radar-\/dev/python/qa\+\_\+ts\+\_\+fft\+\_\+cc.\+py@{/home/erik/prefix/default/src/gr-\/radar-\/dev/python/qa\+\_\+ts\+\_\+fft\+\_\+cc.\+py}}

\begin{DoxyCode}
00001 \textcolor{comment}{#!/usr/bin/env python}
00002 \textcolor{comment}{# -*- coding: utf-8 -*-}
00003 \textcolor{comment}{#}
00004 \textcolor{comment}{# Copyright 2014 Communications Engineering Lab, KIT.}
00005 \textcolor{comment}{#}
00006 \textcolor{comment}{# This is free software; you can redistribute it and/or modify}
00007 \textcolor{comment}{# it under the terms of the GNU General Public License as published by}
00008 \textcolor{comment}{# the Free Software Foundation; either version 3, or (at your option)}
00009 \textcolor{comment}{# any later version.}
00010 \textcolor{comment}{#}
00011 \textcolor{comment}{# This software is distributed in the hope that it will be useful,}
00012 \textcolor{comment}{# but WITHOUT ANY WARRANTY; without even the implied warranty of}
00013 \textcolor{comment}{# MERCHANTABILITY or FITNESS FOR A PARTICULAR PURPOSE.  See the}
00014 \textcolor{comment}{# GNU General Public License for more details.}
00015 \textcolor{comment}{#}
00016 \textcolor{comment}{# You should have received a copy of the GNU General Public License}
00017 \textcolor{comment}{# along with this software; see the file COPYING.  If not, write to}
00018 \textcolor{comment}{# the Free Software Foundation, Inc., 51 Franklin Street,}
00019 \textcolor{comment}{# Boston, MA 02110-1301, USA.}
00020 \textcolor{comment}{#}
00021 
00022 \textcolor{keyword}{from} gnuradio \textcolor{keyword}{import} gr, gr\_unittest
00023 \textcolor{keyword}{from} gnuradio \textcolor{keyword}{import} blocks, fft
00024 \textcolor{keyword}{import} radar\_swig \textcolor{keyword}{as} radar
00025 \textcolor{keyword}{import} numpy \textcolor{keyword}{as} np
00026 \textcolor{keyword}{import} numpy.fft
00027 
00028 \textcolor{keyword}{class }qa_ts_fft_cc (gr\_unittest.TestCase):
00029 
00030     \textcolor{keyword}{def }setUp (self):
00031         self.tb = gr.top\_block ()
00032 
00033     \textcolor{keyword}{def }tearDown (self):
00034         self.tb = \textcolor{keywordtype}{None}
00035 
00036     \textcolor{keyword}{def }test_001_t (self):
00037         \textcolor{comment}{# set up fg}
00038         test\_len = 1024
00039 
00040         packet\_len = test\_len
00041         samp\_rate = 2000
00042         frequency = (100,100)
00043         amplitude = 1
00044 
00045         src = radar.signal\_generator\_cw\_c(packet\_len,samp\_rate,frequency,amplitude)
00046         head = blocks.head(8,test\_len)
00047         fft = radar.ts\_fft\_cc(packet\_len)
00048         snk1 = blocks.vector\_sink\_c()
00049         snk2 = blocks.vector\_sink\_c()
00050 
00051         self.tb.connect(src,head,fft,snk2) \textcolor{comment}{# snk2 holds fft data}
00052         self.tb.connect(head,snk1) \textcolor{comment}{# snk1 holds time samples}
00053         self.tb.run ()
00054         \textcolor{comment}{# check data}
00055         data = snk1.data()
00056         np\_fft = numpy.fft.fft(data) \textcolor{comment}{# get fft}
00057         self.assertComplexTuplesAlmostEqual(snk2.data(),np\_fft,4) \textcolor{comment}{# compare numpy fft and fft from block}
00058 
00059     \textcolor{keyword}{def }test_002_t (self):
00060         \textcolor{comment}{# set up fg}
00061         \textcolor{comment}{# purpse is testing fft on high sample rates}
00062         test\_len = 2**19
00063 
00064         packet\_len = 2**17
00065         min\_output\_buffer = packet\_len*2
00066         samp\_rate = 10000000
00067         frequency = 200000
00068         amplitude = 1
00069 
00070         src = radar.signal\_generator\_cw\_c(packet\_len,samp\_rate,(frequency,frequency),amplitude)
00071         src.set\_min\_output\_buffer(min\_output\_buffer)
00072 
00073         head = blocks.head(8,test\_len)
00074         head.set\_min\_output\_buffer(min\_output\_buffer)
00075 
00076         fft1 = radar.ts\_fft\_cc(packet\_len)
00077         fft1.set\_min\_output\_buffer(min\_output\_buffer)
00078         fft2 = radar.ts\_fft\_cc(packet\_len)
00079         fft2.set\_min\_output\_buffer(min\_output\_buffer)
00080 
00081         snk1 = blocks.vector\_sink\_c()
00082         snk2 = blocks.vector\_sink\_c()
00083 
00084         self.tb.connect(src,head,fft1,snk1)
00085         self.tb.connect(head,fft2,snk2)
00086         self.tb.run ()
00087 
00088         \textcolor{comment}{# check both ffts}
00089         self.assertComplexTuplesAlmostEqual(snk1.data(),snk2.data(),2) \textcolor{comment}{# compare both ffts}
00090 
00091     \textcolor{keyword}{def }test_003_t (self):
00092         \textcolor{comment}{# test fft against gnuradio fft}
00093         \textcolor{comment}{# set up fg}
00094         test\_len = 1024*2
00095 
00096         packet\_len = test\_len
00097         samp\_rate = 2000
00098         frequency = (100,100)
00099         amplitude = 1
00100 
00101         src = radar.signal\_generator\_cw\_c(packet\_len,samp\_rate,frequency,amplitude)
00102         head = blocks.head(8,test\_len)
00103         tsfft = radar.ts\_fft\_cc(packet\_len)
00104         snk1 = blocks.vector\_sink\_c()
00105         self.tb.connect(src,head,tsfft,snk1)
00106 
00107         s2v = blocks.stream\_to\_vector(8, packet\_len)
00108         fft\_inbuild = fft.fft\_vcc(test\_len,\textcolor{keyword}{True},fft.window\_rectangular(0))
00109         snk2 = blocks.vector\_sink\_c()
00110         v2s = blocks.vector\_to\_stream(8, packet\_len);
00111         self.tb.connect(head,s2v,fft\_inbuild,v2s,snk2)
00112 
00113         self.tb.run()
00114 
00115         \textcolor{comment}{# compaire ffts}
00116         data\_tsfft = snk1.data()
00117         data\_fft\_inbuild = snk2.data()
00118 
00119         self.assertComplexTuplesAlmostEqual(data\_tsfft,data\_fft\_inbuild,2) \textcolor{comment}{# compare inbuild fft and fft
       from block}
00120 
00121 \textcolor{keywordflow}{if} \_\_name\_\_ == \textcolor{stringliteral}{'\_\_main\_\_'}:
00122     \textcolor{comment}{#raw\_input('block for gdb',)}
00123     gr\_unittest.run(qa\_ts\_fft\_cc)\textcolor{comment}{#, "qa\_ts\_fft\_cc.xml")}
\end{DoxyCode}
