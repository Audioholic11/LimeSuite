\subsection{Connection\+Novena\+R\+F7\+\_\+streaming.\+cpp}
\label{ConnectionNovenaRF7__streaming_8cpp_source}\index{/home/erik/prefix/default/src/limesuite-\/dev/src/\+Connection\+Novena\+R\+F7/\+Connection\+Novena\+R\+F7\+\_\+streaming.\+cpp@{/home/erik/prefix/default/src/limesuite-\/dev/src/\+Connection\+Novena\+R\+F7/\+Connection\+Novena\+R\+F7\+\_\+streaming.\+cpp}}

\begin{DoxyCode}
00001 \textcolor{preprocessor}{#include "ConnectionNovenaRF7.h"}
00002 
00003 \textcolor{preprocessor}{#include <stdio.h>}
00004 \textcolor{preprocessor}{#include <stdlib.h>}
00005 \textcolor{preprocessor}{#include <fcntl.h>}
00006 \textcolor{preprocessor}{#include <errno.h>}
00007 \textcolor{preprocessor}{#include <string.h>}
00008 \textcolor{preprocessor}{#include <sys/stat.h>}
00009 \textcolor{preprocessor}{#include <fstream>}
00010 \textcolor{preprocessor}{#include <chrono>}
00011 \textcolor{preprocessor}{#include <thread>}
00012 \textcolor{preprocessor}{#include "IConnection.h"}
00013 \textcolor{preprocessor}{#include "ErrorReporting.h"}
00014 
00015 \textcolor{preprocessor}{#if defined(\_\_GNUC\_\_) || defined(\_\_GNUG\_\_)}
00016 \textcolor{preprocessor}{#include <unistd.h>}
00017 \textcolor{preprocessor}{#include <sys/time.h>}
00018 \textcolor{preprocessor}{#endif}
00019 
00020 \textcolor{preprocessor}{#ifdef \_\_linux\_\_}
00021 \textcolor{preprocessor}{#include <sys/mman.h>}
00022 \textcolor{preprocessor}{#include <stdint.h>}
00023 \textcolor{preprocessor}{#include <stdint.h>}
00024 \textcolor{preprocessor}{#include <unistd.h>}
00025 \textcolor{preprocessor}{#include <stdio.h>}
00026 \textcolor{preprocessor}{#include <stdlib.h>}
00027 \textcolor{preprocessor}{#include <getopt.h>}
00028 \textcolor{preprocessor}{#include <fcntl.h>}
00029 \textcolor{preprocessor}{#include <sys/ioctl.h>}
00030 \textcolor{preprocessor}{#include <linux/types.h>}
00031 \textcolor{preprocessor}{#include <linux/spi/spidev.h>}
00032 \textcolor{preprocessor}{#endif}
00033 
00034 \textcolor{preprocessor}{#define IMX6\_EIM\_BASE\_ADDR 0x021b8000}
00035 \textcolor{preprocessor}{#define IMX6\_EIM\_CS0\_BASE (0x00)}
00036 \textcolor{preprocessor}{#define IMX6\_EIM\_CS1\_BASE (0x18)}
00037 \textcolor{preprocessor}{#define IMX6\_EIM\_CS2\_BASE (0x30)}
00038 \textcolor{preprocessor}{#define IMX6\_EIM\_CS3\_BASE (0x48)}
00039 \textcolor{preprocessor}{#define IMX6\_EIM\_WCR (0x90)}
00040 \textcolor{preprocessor}{#define IMX6\_EIM\_WIAR (0x94)}
00041 
00042 \textcolor{preprocessor}{#define IMX6\_EIM\_CS0\_BASE\_ADDR 0x08040000}
00043 \textcolor{preprocessor}{#define IMX6\_EIM\_CS1\_BASE\_ADDR 0x0c040000}
00044 
00045 \textcolor{preprocessor}{#define DATA\_FIFO\_ADDR (IMX6\_EIM\_CS1\_BASE\_ADDR + 0xf000)}
00046 
00047 \textcolor{keyword}{using namespace }std;
00048 \textcolor{keyword}{using namespace }lime;
00049 
00050 \textcolor{keyword}{static} \textcolor{keyword}{const} \textcolor{keywordtype}{int} cSPI_SPEED = 5000000;
00051 
00052 \textcolor{preprocessor}{#ifdef \_\_linux\_\_}
00053 \textcolor{keyword}{static} \textcolor{keywordtype}{int}   *mem\_32 = 0;
00054 \textcolor{keyword}{static} \textcolor{keywordtype}{short} *mem\_16 = 0;
00055 \textcolor{keyword}{static} \textcolor{keywordtype}{char}  *mem\_8 = 0;
00056 \textcolor{keyword}{static} \textcolor{keywordtype}{int}   *prev\_mem\_range = 0;
00057 \textcolor{keyword}{static} \textcolor{keywordtype}{int} mem\_fd = 0;
00058 \textcolor{keyword}{static} \textcolor{keywordtype}{int} fd = 0;
00059 \textcolor{preprocessor}{#endif}
00060 
00061 \textcolor{keywordtype}{bool} eim_configured = \textcolor{keyword}{false};
00062 
00063 \textcolor{keywordtype}{int} fpga_send(\textcolor{keywordtype}{unsigned} \textcolor{keyword}{const} \textcolor{keywordtype}{int} addr, \textcolor{keyword}{const} \textcolor{keywordtype}{char}* buf, \textcolor{keyword}{const} \textcolor{keywordtype}{int} len)
00064 \{
00065 \textcolor{preprocessor}{#ifdef \_\_linux\_\_}
00066     \textcolor{keywordtype}{void}* mem\_base = 0;
00067     \textcolor{keywordflow}{if}(mem\_32)
00068         munmap(mem\_32, 0xFFFF);
00069     \textcolor{keywordflow}{if}(fd)
00070         close(fd);
00071 
00072     fd = open(\textcolor{stringliteral}{"/dev/mem"}, O\_RDWR | O\_SYNC);
00073     \textcolor{keywordflow}{if}( fd < 0 )
00074     \{
00075         perror(\textcolor{stringliteral}{"Unable to open /dev/mem"});
00076         fd = 0;
00077         \textcolor{keywordflow}{return} -1;
00078     \}
00079 
00080     \textcolor{keyword}{const} \textcolor{keywordtype}{unsigned} \textcolor{keywordtype}{int} map\_size = 4096;
00081     \textcolor{keyword}{const} \textcolor{keywordtype}{unsigned} \textcolor{keywordtype}{int} map\_mask = map\_size-1;
00082 
00083     mem\_base = mmap(0, map\_size, PROT\_READ | PROT\_WRITE, MAP\_SHARED, fd, addr & ~map\_mask);
00084     \textcolor{keywordflow}{if}(mem\_base < 0)
00085     \{
00086         printf(\textcolor{stringliteral}{"error mapping memory\(\backslash\)n"});
00087         \textcolor{keywordflow}{return} -1;
00088     \}
00089 
00090     \textcolor{keywordtype}{size\_t} virt\_addr = size\_t(mem\_base) + (size\_t(addr) & map\_mask);
00091     \textcolor{keywordflow}{for}(\textcolor{keywordtype}{unsigned} i=0; i<len/\textcolor{keyword}{sizeof}(\textcolor{keywordtype}{unsigned} short); ++i)
00092     \{
00093         *(\textcolor{keywordtype}{unsigned} \textcolor{keywordtype}{short}*)(virt\_addr+i*2) = ((\textcolor{keywordtype}{unsigned} \textcolor{keywordtype}{short}*)buf)[i];
00094     \}
00095     \textcolor{keywordflow}{if}(munmap(mem\_base, map\_size) == -1)
00096         \textcolor{keywordflow}{return} -2;
00097     close(fd);
00098 \textcolor{preprocessor}{#endif}
00099     \textcolor{keywordflow}{return} 0;
00100 \}
00101 
00102 \textcolor{keywordtype}{int} fpga_read(\textcolor{keywordtype}{unsigned} \textcolor{keyword}{const} \textcolor{keywordtype}{int} addr, \textcolor{keywordtype}{unsigned} \textcolor{keywordtype}{short} *buf, \textcolor{keyword}{const} \textcolor{keywordtype}{int} len)
00103 \{
00104 \textcolor{preprocessor}{#ifdef \_\_linux\_\_}
00105     \textcolor{keywordtype}{void}* mem\_base = 0;
00106 
00107     \textcolor{keywordflow}{if}(mem\_32)
00108         munmap(mem\_32, 0xFFFF);
00109     \textcolor{keywordflow}{if}(fd)
00110         close(fd);
00111 
00112     fd = open(\textcolor{stringliteral}{"/dev/mem"}, O\_RDWR | O\_SYNC);
00113     \textcolor{keywordflow}{if}( fd < 0 )
00114     \{
00115         perror(\textcolor{stringliteral}{"Unable to open /dev/mem"});
00116         fd = 0;
00117         \textcolor{keywordflow}{return} -1;
00118     \}
00119 
00120     \textcolor{keyword}{const} \textcolor{keywordtype}{unsigned} \textcolor{keywordtype}{int} map\_size = 4096;
00121     \textcolor{keyword}{const} \textcolor{keywordtype}{unsigned} \textcolor{keywordtype}{int} map\_mask = map\_size-1;
00122 
00123     mem\_base = mmap(0, map\_size, PROT\_READ | PROT\_WRITE, MAP\_SHARED, fd, addr & ~map\_mask);
00124     \textcolor{comment}{//mem\_base = ioremap\_wc(addr & ~map\_mask, map\_size);}
00125     \textcolor{keywordflow}{if}(mem\_base < 0)
00126     \{
00127         printf(\textcolor{stringliteral}{"error mapping memory\(\backslash\)n"});
00128         \textcolor{keywordflow}{return} -1;
00129     \}
00130     \textcolor{keywordtype}{size\_t} virt\_addr = size\_t(mem\_base) + (size\_t(addr) & map\_mask);
00131     memcpy(buf, (\textcolor{keywordtype}{void} *)virt\_addr, len);
00132     \textcolor{keywordflow}{if}(munmap(mem\_base, map\_size) == -1)
00133     \{
00134         printf(\textcolor{stringliteral}{"munmap failed\(\backslash\)n"});
00135         \textcolor{keywordflow}{return} -2;
00136     \}
00137     close(fd);
00138 \textcolor{preprocessor}{#endif}
00139     \textcolor{keywordflow}{return} 0;
00140 \}
00141 
00142 \textcolor{keywordtype}{int} readKernelMemory(\textcolor{keywordtype}{long} offset, \textcolor{keywordtype}{int} virtualized, \textcolor{keywordtype}{int} size)
00143 \{
00144 \textcolor{preprocessor}{#ifdef \_\_linux\_\_}
00145     \textcolor{keywordtype}{int} result(0);
00146 
00147     \textcolor{keywordtype}{int} *mem\_range = (\textcolor{keywordtype}{int} *)(offset & ~0xFFFF);
00148     \textcolor{keywordflow}{if} (mem\_range != prev\_mem\_range)
00149     \{
00150         prev\_mem\_range = mem\_range;
00151 
00152         \textcolor{keywordflow}{if} (mem\_32)
00153             munmap(mem\_32, 0xFFFF);
00154         \textcolor{keywordflow}{if} (mem\_fd)
00155             close(mem\_fd);
00156 
00157         \textcolor{keywordflow}{if} (virtualized)
00158         \{
00159             mem\_fd = open(\textcolor{stringliteral}{"/dev/kmem"}, O\_RDWR);
00160             \textcolor{keywordflow}{if} (mem\_fd < 0)
00161             \{
00162                 printf(\textcolor{stringliteral}{"%s %s\(\backslash\)n"}, \textcolor{stringliteral}{"Couldn't open /dev/mem"}, strerror(errno));
00163                 mem\_fd = 0;
00164                 \textcolor{keywordflow}{return} -1;
00165             \}
00166         \}
00167         \textcolor{keywordflow}{else}
00168         \{
00169             mem\_fd = open(\textcolor{stringliteral}{"/dev/mem"}, O\_RDWR);
00170             \textcolor{keywordflow}{if} (mem\_fd < 0)
00171             \{
00172                 printf(\textcolor{stringliteral}{"%s %s\(\backslash\)n"}, \textcolor{stringliteral}{"Couldn't open /dev/mem"}, strerror(errno));
00173                 mem\_fd = 0;
00174                 \textcolor{keywordflow}{return} -1;
00175             \}
00176         \}
00177 
00178         mem\_32 = (\textcolor{keywordtype}{int} *)mmap(0, 0xffff, PROT\_READ | PROT\_WRITE, MAP\_SHARED, mem\_fd, offset&~0xFFFF);
00179         \textcolor{keywordflow}{if} ((\textcolor{keywordtype}{void} *)-1 == mem\_32)
00180         \{
00181             printf(\textcolor{stringliteral}{"%s %s\(\backslash\)n"}, \textcolor{stringliteral}{"Couldn't mmap /dev/kmem: "}, strerror(errno));
00182 
00183             \textcolor{keywordflow}{if} (-1 == close(mem\_fd))
00184                 printf(\textcolor{stringliteral}{"%s %s\(\backslash\)n"}, \textcolor{stringliteral}{"Also couldn't close /dev/kmem: "}, strerror(errno));
00185 
00186             mem\_fd = 0;
00187             \textcolor{keywordflow}{return} -1;
00188         \}
00189         mem\_16 = (\textcolor{keywordtype}{short} *)mem\_32;
00190         mem\_8 = (\textcolor{keywordtype}{char} *)mem\_32;
00191     \}
00192 
00193     \textcolor{keywordtype}{int} scaled\_offset = (offset - (offset&~0xFFFF));
00194     \textcolor{keywordflow}{if} (size == 1)
00195     \{
00196         \textcolor{keywordflow}{if} (mem\_8)
00197             result = mem\_8[scaled\_offset / \textcolor{keyword}{sizeof}(char)];
00198     \}
00199     \textcolor{keywordflow}{else} \textcolor{keywordflow}{if} (size == 2)
00200     \{
00201         \textcolor{keywordflow}{if} (mem\_16)
00202             result = mem\_16[scaled\_offset / \textcolor{keyword}{sizeof}(short)];
00203     \}
00204     \textcolor{keywordflow}{else}
00205     \{
00206         \textcolor{keywordflow}{if} (mem\_32)
00207             result = mem\_32[scaled\_offset / \textcolor{keyword}{sizeof}(long)];
00208     \}
00209 
00210     \textcolor{keywordflow}{return} result;
00211 \textcolor{preprocessor}{#else}
00212     \textcolor{keywordflow}{return} 0;
00213 \textcolor{preprocessor}{#endif}
00214 \}
00215 
00216 \textcolor{keywordtype}{int} writeKernelMemory(\textcolor{keywordtype}{long} offset, \textcolor{keywordtype}{long} value, \textcolor{keywordtype}{int} virtualized, \textcolor{keywordtype}{int} size)
00217 \{
00218 \textcolor{preprocessor}{#ifdef \_\_linux\_\_}
00219     \textcolor{keywordtype}{int} old\_value = readKernelMemory(offset, virtualized, size);
00220     \textcolor{keywordtype}{int} scaled\_offset = (offset - (offset&~0xFFFF));
00221     \textcolor{keywordflow}{if} (size == 1)
00222     \{
00223         \textcolor{keywordflow}{if} (mem\_8)
00224             mem\_8[scaled\_offset / \textcolor{keyword}{sizeof}(char)] = value;
00225     \}
00226     \textcolor{keywordflow}{else} \textcolor{keywordflow}{if} (size == 2)
00227     \{
00228         \textcolor{keywordflow}{if} (mem\_16)
00229             mem\_16[scaled\_offset / \textcolor{keyword}{sizeof}(short)] = value;
00230     \}
00231     \textcolor{keywordflow}{else}
00232     \{
00233         \textcolor{keywordflow}{if} (mem\_32)
00234             mem\_32[scaled\_offset / \textcolor{keyword}{sizeof}(long)] = value;
00235     \}
00236     \textcolor{keywordflow}{return} old\_value;
00237 \textcolor{preprocessor}{#endif // \_\_linux\_\_}
00238     \textcolor{keywordflow}{return} 0;
00239 \}
00240 
00241 \textcolor{keywordtype}{int} prep_eim_burst()
00242 \{
00243     \textcolor{keywordflow}{if}(eim_configured == \textcolor{keyword}{true})
00244         \textcolor{keywordflow}{return} 1;
00245     eim_configured = \textcolor{keyword}{true};
00246     \textcolor{comment}{// set up pads to be mapped to EIM}
00247     \textcolor{keywordflow}{for} (\textcolor{keywordtype}{int} i = 0; i < 16; i++)
00248     \{
00249         writeKernelMemory(0x20e0114 + i*4, 0x0, 0, 4); \textcolor{comment}{// mux mapping}
00250         writeKernelMemory(0x20e0428 + i*4, 0xb0b1, 0, 4); \textcolor{comment}{// pad strength config'd for a 100MHz rate}
00251     \}
00252 
00253     \textcolor{comment}{// mux mapping}
00254     writeKernelMemory(0x20e046c - 0x314, 0x0, 0, 4); \textcolor{comment}{// BCLK}
00255     writeKernelMemory(0x20e040c - 0x314, 0x0, 0, 4); \textcolor{comment}{// CS0}
00256     writeKernelMemory(0x20e0410 - 0x314, 0x0, 0, 4); \textcolor{comment}{// CS1}
00257     writeKernelMemory(0x20e0414 - 0x314, 0x0, 0, 4); \textcolor{comment}{// OE}
00258     writeKernelMemory(0x20e0418 - 0x314, 0x0, 0, 4); \textcolor{comment}{// RW}
00259     writeKernelMemory(0x20e041c - 0x314, 0x0, 0, 4); \textcolor{comment}{// LBA}
00260     writeKernelMemory(0x20e0468 - 0x314, 0x0, 0, 4); \textcolor{comment}{// WAIT}
00261     writeKernelMemory(0x20e0408 - 0x314, 0x0, 0, 4); \textcolor{comment}{// A16}
00262     writeKernelMemory(0x20e0404 - 0x314, 0x0, 0, 4); \textcolor{comment}{// A17}
00263     writeKernelMemory(0x20e0400 - 0x314, 0x0, 0, 4); \textcolor{comment}{// A18}
00264     writeKernelMemory(0x20e0400 - 0x314, 0x0, 0, 4); \textcolor{comment}{// A18}
00265 
00266     \textcolor{comment}{// pad strength}
00267     writeKernelMemory(0x20e046c, 0xb0b1, 0, 4); \textcolor{comment}{// BCLK}
00268     writeKernelMemory(0x20e040c, 0xb0b1, 0, 4); \textcolor{comment}{// CS0}
00269     writeKernelMemory(0x20e0410, 0xb0b1, 0, 4); \textcolor{comment}{// CS1}
00270     writeKernelMemory(0x20e0414, 0xb0b1, 0, 4); \textcolor{comment}{// OE}
00271     writeKernelMemory(0x20e0418, 0xb0b1, 0, 4); \textcolor{comment}{// RW}
00272     writeKernelMemory(0x20e041c, 0xb0b1, 0, 4); \textcolor{comment}{// LBA}
00273     writeKernelMemory(0x20e0468, 0xb0b1, 0, 4); \textcolor{comment}{// WAIT}
00274     writeKernelMemory(0x20e0408, 0xb0b1, 0, 4); \textcolor{comment}{// A16}
00275     writeKernelMemory(0x20e0404, 0xb0b1, 0, 4); \textcolor{comment}{// A17}
00276     writeKernelMemory(0x20e0400, 0xb0b1, 0, 4); \textcolor{comment}{// A18}
00277 
00278     writeKernelMemory(0x020c4080, 0xcf3, 0, 4); \textcolor{comment}{// ungate eim slow clocks}
00279 
00280     \textcolor{comment}{// EIM\_CSnGCR1}
00281     \textcolor{comment}{// rework timing for sync use}
00282     \textcolor{comment}{// PSZ WP GBC AUS CSREC SP DSZ BCS BCD WC BL CREP CRE RFL WFL MUM SRD SWR CSEN}
00283 
00284     \textcolor{keywordtype}{int} PSZ = 1 << 28; \textcolor{comment}{// 2048 words page size}
00285     \textcolor{keywordtype}{int} WP = 0 << 27; \textcolor{comment}{//(not protected)}
00286     \textcolor{keywordtype}{int} GBC = 1 << 24; \textcolor{comment}{//min 1 cycles between chip select changes}
00287     \textcolor{keywordtype}{int} AUS = 0 << 23; \textcolor{comment}{//0 address shifted according to port size / 1: unshifted}
00288     \textcolor{keywordtype}{int} CSREC = 1 << 20; \textcolor{comment}{//min 1 cycles between CS, OE, WE signals}
00289     \textcolor{keywordtype}{int} SP = 0 << 19; \textcolor{comment}{//no supervisor protect (user mode access allowed)}
00290     \textcolor{keywordtype}{int} DSZ = 1 << 16; \textcolor{comment}{//16-bit port resides on DATA[15:0]}
00291     \textcolor{keywordtype}{int} BCS = 0 << 14; \textcolor{comment}{//0 clock delay for burst generation}
00292     \textcolor{keywordtype}{int} BCD = 0 << 12; \textcolor{comment}{//divide EIM clock by 1 for burst clock}
00293     \textcolor{keywordtype}{int} WC = 1 << 11; \textcolor{comment}{//write accesses are continuous burst length}
00294     \textcolor{keywordtype}{int} BL = 2 << 8; \textcolor{comment}{//32 word memory wrap length}
00295     \textcolor{keywordtype}{int} CREP = 1 << 7; \textcolor{comment}{//non-PSRAM, set to 1}
00296     \textcolor{keywordtype}{int} CRE = 0 << 6; \textcolor{comment}{//CRE is disabled}
00297     \textcolor{keywordtype}{int} RFL = 1 << 5; \textcolor{comment}{//fixed latency reads}
00298     \textcolor{keywordtype}{int} WFL = 1 << 4; \textcolor{comment}{//fixed latency writes}
00299     \textcolor{keywordtype}{int} MUM = 1 << 3; \textcolor{comment}{//multiplexed mode enabled}
00300     \textcolor{keywordtype}{int} SRD = 1 << 2; \textcolor{comment}{//synch reads}
00301     \textcolor{keywordtype}{int} SWR = 1 << 1; \textcolor{comment}{//synch writes}
00302     \textcolor{keywordtype}{int} CSEN = 1; \textcolor{comment}{//chip select is enabled}
00303     \textcolor{keywordtype}{int} EIM\_CSnGCR1 = PSZ|WP|GBC|AUS|CSREC|SP|DSZ|BCS|BCD|WC|BL|CREP|CRE|RFL|WFL|MUM|SRD|SWR|CSEN;
00304     printf(\textcolor{stringliteral}{"EIM\_CSnGCR1 0x%08X\(\backslash\)n"}, EIM\_CSnGCR1);
00305 
00306     writeKernelMemory(0x21b8000, EIM\_CSnGCR1, 0, 4);
00307     writeKernelMemory(0x21b8000+24, EIM\_CSnGCR1, 0, 4);
00308     writeKernelMemory(0x21b8000+48, EIM\_CSnGCR1, 0, 4);
00309 
00310     \textcolor{comment}{// EIM\_CS0GCR2}
00311     \textcolor{keywordtype}{int} MUX16\_BYP\_GRANT = 1 << 12;
00312     \textcolor{keywordtype}{int} DAP = 0 << 9;
00313     \textcolor{keywordtype}{int} DAE = 0 << 8;
00314     \textcolor{keywordtype}{int} DAPS = 0 << 4;
00315     \textcolor{keywordtype}{int} ADH = 0; \textcolor{comment}{// address hold time after ADC negation(0 cycles)}
00316     \textcolor{keywordtype}{int} EIM\_CSnGCR2 = MUX16\_BYP\_GRANT|DAP|DAE|DAPS|ADH;
00317     printf(\textcolor{stringliteral}{"EIM\_CSnGCR2 0x%08X\(\backslash\)n"}, EIM\_CSnGCR2);
00318     writeKernelMemory(0x21b8004, EIM\_CSnGCR2, 0, 4);
00319     writeKernelMemory(0x21b8004+24, EIM\_CSnGCR2, 0, 4);
00320     writeKernelMemory(0x21b8004+48, EIM\_CSnGCR2, 0, 4);
00321 
00322     \textcolor{comment}{// EIM\_CS0RCR1}
00323     \textcolor{comment}{// RWSC RADVA RAL RADVN OEA OEN RCSA RCSN}
00324     \textcolor{keywordtype}{int} RWSC = 9 << 24;
00325     \textcolor{keywordtype}{int} RADVA = 0 << 20;
00326     \textcolor{keywordtype}{int} RAL = 0 << 19;
00327     \textcolor{keywordtype}{int} RADVN = 1 << 16;
00328     \textcolor{keywordtype}{int} OEA = 4 << 12;
00329     \textcolor{keywordtype}{int} OEN = 0 << 8;
00330     \textcolor{keywordtype}{int} RCSA = 0 << 4;
00331     \textcolor{keywordtype}{int} RCSN = 0;
00332     \textcolor{keywordtype}{int} EIM\_CSnRCR1 = RWSC|RADVA|RAL|RADVN|OEA|OEN|RCSA|RCSN;
00333     writeKernelMemory(0x21b8008, EIM\_CSnRCR1, 0, 4);
00334     writeKernelMemory(0x21b8008+24, EIM\_CSnRCR1, 0, 4);
00335     writeKernelMemory(0x21b8008+48, EIM\_CSnRCR1, 0, 4);
00336     printf(\textcolor{stringliteral}{"EIM\_CSnRCR1 0x%08X\(\backslash\)n"}, EIM\_CSnRCR1);
00337 
00338     \textcolor{comment}{// EIM\_CS0RCR2}
00339     \textcolor{comment}{// APR PAT RL RBEA RBEN}
00340     \textcolor{keywordtype}{int} APR = 0 << 15; \textcolor{comment}{// 0 mandatory because MUM = 1}
00341     \textcolor{keywordtype}{int} PAT = 0 << 12; \textcolor{comment}{// XXX because APR = 0}
00342     \textcolor{keywordtype}{int} RL = 0 << 8; \textcolor{comment}{//}
00343     \textcolor{keywordtype}{int} RBEA = 0 << 4; \textcolor{comment}{//these match RCSA/RCSN from previous field}
00344     \textcolor{keywordtype}{int} RBE = 0 << 3;
00345     \textcolor{keywordtype}{int} RBEN = 0;
00346     \textcolor{keywordtype}{int} EIM\_CSnRCR2 = APR|PAT|RL|RBEA|RBE|RBEN;
00347     writeKernelMemory(0x21b800c, EIM\_CSnRCR2, 0, 4);
00348     writeKernelMemory(0x21b800c+24, EIM\_CSnRCR2, 0, 4);
00349     writeKernelMemory(0x21b800c+48, EIM\_CSnRCR2, 0, 4);
00350     printf(\textcolor{stringliteral}{"EIM\_CSnRCR2 0x%08X\(\backslash\)n"}, EIM\_CSnRCR2);
00351 
00352     \textcolor{comment}{// EIM\_CS0WCR1}
00353     \textcolor{comment}{// WAL WBED WWSC WADVA WADVN WBEA WBEN WEA WEN WCSA WCSN}
00354     \textcolor{keywordtype}{int} WAL = 0 << 31; \textcolor{comment}{//use WADVN}
00355     \textcolor{keywordtype}{int} WBED = 0 << 30; \textcolor{comment}{//allow BE during write}
00356     \textcolor{keywordtype}{int} WWSC = 9 << 24; \textcolor{comment}{// write wait states}
00357     \textcolor{keywordtype}{int} WADVA = 0 << 21; \textcolor{comment}{//same as RADVA}
00358     \textcolor{keywordtype}{int} WADVN = 2 << 18; \textcolor{comment}{//this sets WE length to 1 (this value +1)}
00359     \textcolor{keywordtype}{int} WBEA = 0 << 15; \textcolor{comment}{//same as RBEA}
00360     \textcolor{keywordtype}{int} WBEN = 0 << 12; \textcolor{comment}{//same as RBEN}
00361     \textcolor{keywordtype}{int} WEA = 4 << 9; \textcolor{comment}{//0 cycles between beginning of access and WE assertion}
00362     \textcolor{keywordtype}{int} WEN = 0 << 6; \textcolor{comment}{//1 cycles to end of WE assertion}
00363     \textcolor{keywordtype}{int} WCSA = 0 << 3; \textcolor{comment}{//cycles to CS assertion}
00364     \textcolor{keywordtype}{int} WCSN = 0 ; \textcolor{comment}{//cycles to CS negation}
00365     \textcolor{keywordtype}{int} EIM\_CSnWCR1 = WAL|WBED|WWSC|WADVA|WADVN|WBEA|WBEN|WEA|WEN|WCSA|WCSN;
00366     printf(\textcolor{stringliteral}{"EIM\_CSnWCR1 0x%08X\(\backslash\)n"}, EIM\_CSnWCR1);
00367     writeKernelMemory(0x21b8010, EIM\_CSnWCR1, 0, 4);
00368     writeKernelMemory(0x21b8010+24, EIM\_CSnWCR1, 0, 4); \textcolor{comment}{//cs1}
00369     writeKernelMemory(0x21b8010+48, EIM\_CSnWCR1, 0, 4); \textcolor{comment}{//cs2}
00370 
00371     \textcolor{comment}{// EIM\_WCR}
00372     \textcolor{keywordtype}{int} WDOG\_LIMIT = 3 << 9;
00373     \textcolor{keywordtype}{int} WDOG\_EN = 1 << 8;
00374     \textcolor{keywordtype}{int} INTPOL = 0 << 5;
00375     \textcolor{keywordtype}{int} INTEN = 0 << 4;
00376     \textcolor{keywordtype}{int} GBCD = 0 << 1; \textcolor{comment}{//0 //don't divide the burst clock}
00377     \textcolor{keywordtype}{int} BCM = 1; \textcolor{comment}{//free-run BCLK}
00378     \textcolor{keywordtype}{int} EIM\_WCR = WDOG\_LIMIT|WDOG\_EN|INTPOL|INTEN|GBCD|BCM;
00379     writeKernelMemory(0x21b8090, EIM\_WCR, 0, 4);
00380     printf(\textcolor{stringliteral}{"EIM\_WCR 0x%08X\(\backslash\)n"}, EIM\_WCR);
00381 
00382     \textcolor{comment}{// EIM\_WIAR}
00383     \textcolor{keywordtype}{int} ACLK\_EN = 1 << 4;
00384     \textcolor{keywordtype}{int} ERRST = 0 << 3;
00385     \textcolor{keywordtype}{int} INT = 0 << 2;
00386     \textcolor{keywordtype}{int} IPS\_ACK = 0 << 1;
00387     \textcolor{keywordtype}{int} IPS\_REQ = 0;
00388     \textcolor{keywordtype}{int} EIM\_WIAR = ACLK\_EN|ERRST|INT|IPS\_ACK|IPS\_REQ;
00389     writeKernelMemory(0x21b8094, EIM\_WIAR, 0, 4);
00390     printf(\textcolor{stringliteral}{"EIM\_WIAR 0x%08X\(\backslash\)n"}, EIM\_WIAR);
00391 
00392     printf( \textcolor{stringliteral}{"resetting CS0 space to 64M and enabling 32M CS1 and 32M CS2 space.\(\backslash\)n"} );
00393     writeKernelMemory( 0x20e0004, (readKernelMemory(0x20e0004, 0, 4) & 0xFFFFF000) | 0x04B, 0, 4);
00394     printf(\textcolor{stringliteral}{"EIM configured\(\backslash\)n"});
00395     \textcolor{keywordflow}{return} 0;
00396 \}
00397 
00398 \textcolor{keywordtype}{void} ConnectionNovenaRF7::ReceivePacketsLoop(\textcolor{keyword}{const} ThreadData args)
00399 \{
00400     prep_eim_burst();
00401     ConnectionNovenaRF7 *pthis = args.dataPort;
00402     \textcolor{keyword}{auto} terminate = args.terminate;
00403     \textcolor{keyword}{auto} dataRate\_Bps = args.dataRate_Bps;
00404 
00405     uint32\_t samplesCollected = 0;
00406     \textcolor{keyword}{auto} t1 = chrono::high\_resolution\_clock::now();
00407     \textcolor{keyword}{auto} t2 = chrono::high\_resolution\_clock::now();
00408 
00409     \textcolor{keyword}{const} \textcolor{keywordtype}{int} buffer\_size = 32768*2;
00410     \textcolor{keywordtype}{char} buffer[buffer\_size];
00411     memset(buffer, 0, buffer\_size);
00412 
00413     \textcolor{keywordtype}{unsigned} \textcolor{keywordtype}{long} totalBytesReceived = 0;
00414     \textcolor{keywordtype}{int} m\_bufferFailures = 0;
00415     int16\_t sample;
00416 
00417     \textcolor{keyword}{const} \textcolor{keywordtype}{int} bytesToRead = 4096;
00418     \textcolor{keyword}{const} \textcolor{keywordtype}{int} FPGAbufferSize = 32768*2;
00419 
00420     \textcolor{comment}{//int dataSource = 0;}
00421     \textcolor{keyword}{const} uint16\_t NOVENA\_DATA\_SRC\_ADDR = 0x0702;
00422     uint16\_t controlRegValue = 0;
00423     pthis->ReadRegister(NOVENA\_DATA\_SRC\_ADDR, controlRegValue);
00424 
00425     \textcolor{comment}{//dataSource = (controlRegValue >> 12) & 0x3;}
00426     \textcolor{comment}{//reset FIFO}
00427     pthis->WriteRegister(NOVENA\_DATA\_SRC\_ADDR, (controlRegValue & 0x7FFF));
00428     pthis->WriteRegister(NOVENA\_DATA\_SRC\_ADDR, (controlRegValue & 0x7FFF) | 0x8000);
00429     pthis->WriteRegister(NOVENA\_DATA\_SRC\_ADDR, (controlRegValue & 0x7FFF));
00430     \textcolor{comment}{//set data source}
00431     \textcolor{comment}{//pthis->Reg\_write(NOVENA\_DATA\_SRC\_ADDR, (controlRegValue & 0x8FFF) | (dataSource << 12));}
00432     \textcolor{comment}{//request data}
00433     pthis->WriteRegister(NOVENA\_DATA\_SRC\_ADDR, (controlRegValue & 0xBFFF));
00434     pthis->WriteRegister(NOVENA\_DATA\_SRC\_ADDR, (controlRegValue & 0xBFFF)| 0x4000);
00435 
00436     vector<complex16\_t> samples(FPGAbufferSize/4);
00437     uint64\_t timestamp = 0;
00438     uint32\_t droppedSamples = 0;
00439 
00440     \textcolor{keywordflow}{while} (terminate->load() == \textcolor{keyword}{false})
00441     \{
00442         std::this\_thread::sleep\_for(std::chrono::milliseconds(2));
00443 
00444         \textcolor{keywordtype}{int} bytesReceived = 0;
00445 \textcolor{preprocessor}{#ifndef NDEBUG}
00446         printf(\textcolor{stringliteral}{"--- FPGA FIFO UPDATE ---\(\backslash\)n"});
00447 \textcolor{preprocessor}{#endif}
00448         \textcolor{keywordflow}{for} (\textcolor{keywordtype}{int} bb = 0; bb<FPGAbufferSize; bb += bytesToRead)
00449         \{
00450             fpga_read(0xC000000, (\textcolor{keywordtype}{unsigned} \textcolor{keywordtype}{short}*)&buffer[bb], bytesToRead);
00451             totalBytesReceived += bytesToRead;
00452             bytesReceived += bytesToRead;
00453         \}
00454     \textcolor{comment}{//reset FIFO}
00455         pthis->WriteRegister(NOVENA\_DATA\_SRC\_ADDR, (controlRegValue & 0x7FFF));
00456         pthis->WriteRegister(NOVENA\_DATA\_SRC\_ADDR, (controlRegValue & 0x7FFF) | 0x8000);
00457         pthis->WriteRegister(NOVENA\_DATA\_SRC\_ADDR, (controlRegValue & 0x7FFF));
00458         \textcolor{comment}{//request data}
00459         pthis->WriteRegister(NOVENA\_DATA\_SRC\_ADDR, (controlRegValue & 0xBFFF));
00460         pthis->WriteRegister(NOVENA\_DATA\_SRC\_ADDR, (controlRegValue & 0xBFFF)| 0x4000);
00461 
00462         \textcolor{keywordflow}{if} (bytesReceived > 0)
00463         \{
00464             \textcolor{keywordtype}{char}* bufStart = buffer;
00465             \textcolor{keywordflow}{for} (\textcolor{keywordtype}{int} p = 0; p < bytesReceived; p += 4)
00466             \{
00467                 sample = (bufStart[p + 1] & 0x0F);
00468                 sample = sample << 8;
00469                 sample |= (bufStart[p] & 0xFF);
00470                 sample = sample << 4;
00471                 sample = sample >> 4;
00472                 samples[samplesCollected].i = sample;
00473 
00474                 sample = (bufStart[p + 1] & 0x0F);
00475                 sample = sample << 8;
00476                 sample |= (bufStart[p] & 0xFF);
00477                 sample = sample << 4;
00478                 sample = sample >> 4;
00479                 samples[samplesCollected].q = sample;
00480 
00481                 ++samplesCollected;
00482             \}
00483 
00484             IStreamChannel::Metadata meta;
00485             meta.timestamp = timestamp;
00486             timestamp += samplesCollected;
00487             meta.flags = 0;
00488             uint32\_t samplesPushed = args.channels[0]->Write((\textcolor{keyword}{const} \textcolor{keywordtype}{void}*)samples.data(), samplesCollected,
       &meta, 100);
00489             \textcolor{keywordflow}{if}(samplesPushed != samplesCollected)
00490                 droppedSamples += samplesCollected-samplesPushed;
00491             samplesCollected = 0;
00492         \}
00493         \textcolor{keywordflow}{else}
00494         \{
00495             ++m\_bufferFailures;
00496         \}
00497 
00498         t2 = chrono::high\_resolution\_clock::now();
00499         \textcolor{keywordtype}{long} timePeriod = std::chrono::duration\_cast<std::chrono::milliseconds>(
      t2 - t1).count();
00500         \textcolor{keywordflow}{if} (timePeriod >= 1000)
00501         \{
00502             \textcolor{keywordtype}{float} m\_dataRate = 1000.0*totalBytesReceived / timePeriod;
00503             t1 = t2;
00504             totalBytesReceived = 0;
00505 
00506             \textcolor{keywordflow}{if}(dataRate\_Bps)
00507                 dataRate\_Bps->store(m\_dataRate);
00508 \textcolor{preprocessor}{#ifndef NDEBUG}
00509             printf(\textcolor{stringliteral}{"Rx: %.0f kB/s failures:%i\(\backslash\)n"}, m\_dataRate / 1000.0, m\_bufferFailures);
00510 \textcolor{preprocessor}{#endif}
00511             m\_bufferFailures = 0;
00512             droppedSamples = 0;
00513         \}
00514         memset(buffer, 0, buffer\_size);
00515     \}
00516 \textcolor{preprocessor}{#ifndef NDEBUG}
00517     printf(\textcolor{stringliteral}{"Rx finished\(\backslash\)n"});
00518 \textcolor{preprocessor}{#endif}
00519 \}
00520 
00521 \textcolor{keywordtype}{int} ConnectionNovenaRF7::ReadStream(\textcolor{keyword}{const} \textcolor{keywordtype}{size\_t} streamID, \textcolor{keywordtype}{void}* buffs, \textcolor{keyword}{const} \textcolor{keywordtype}{size\_t} 
      length, \textcolor{keyword}{const} \textcolor{keywordtype}{long} timeout_ms, StreamMetadata &metadata)
00522 \{
00523     assert(streamID != 0);
00524     lime::IStreamChannel* channel = (lime::IStreamChannel*)streamID;
00525     lime::IStreamChannel::Metadata meta;
00526     meta.flags = 0;
00527     meta.flags |= metadata.hasTimestamp ? lime::IStreamChannel::Metadata::SYNC\_TIMESTAMP : 0;
00528     meta.timestamp = metadata.timestamp;
00529     \textcolor{keywordtype}{int} status = channel->Read(buffs, length, &meta, timeout\_ms);
00530     metadata.timestamp = meta.timestamp;
00531     \textcolor{keywordflow}{return} status;
00532 \}
00533 
00534 \textcolor{keywordtype}{int} ConnectionNovenaRF7::WriteStream(\textcolor{keyword}{const} \textcolor{keywordtype}{size\_t} streamID, \textcolor{keyword}{const} \textcolor{keywordtype}{void}* buffs, \textcolor{keyword}{const} \textcolor{keywordtype}{size\_t} 
      length, \textcolor{keyword}{const} \textcolor{keywordtype}{long} timeout_ms, \textcolor{keyword}{const} StreamMetadata &metadata)
00535 \{
00536     \textcolor{keywordflow}{return} -1;
00537 \}
00538 
00539 \textcolor{keywordtype}{int} ConnectionNovenaRF7::SetupStream(\textcolor{keywordtype}{size\_t} &streamID, \textcolor{keyword}{const} StreamConfig &config)
00540 \{
00541     \textcolor{keywordflow}{if}(rxRunning.load() == \textcolor{keyword}{true})
00542         \textcolor{keywordflow}{return} ReportError(EPERM, \textcolor{stringliteral}{"All streams must be stopped before doing setups"});
00543     streamID = ~0;
00544     StreamChannel* stream = \textcolor{keyword}{new} StreamChannel(\textcolor{keyword}{this});
00545     stream->config = config;
00546     \textcolor{comment}{//TODO check for duplicate streams}
00547     \textcolor{keywordflow}{if}(config.isTx)
00548         \textcolor{keywordflow}{return} ReportError(ENOTSUP, \textcolor{stringliteral}{"Transmitting not supported"});
00549     \textcolor{keywordflow}{else}
00550         mRxStreams.push\_back(stream);
00551     streamID = size\_t(stream);
00552     \textcolor{keywordflow}{return} 0; \textcolor{comment}{//success}
00553 \}
00554 
00555 \textcolor{keywordtype}{int} ConnectionNovenaRF7::CloseStream(\textcolor{keyword}{const} \textcolor{keywordtype}{size\_t} streamID)
00556 \{
00557     \textcolor{keywordflow}{if}(rxRunning.load() == \textcolor{keyword}{true})
00558         \textcolor{keywordflow}{return} ReportError(EPERM, \textcolor{stringliteral}{"All streams must be stopped before closing"});
00559     StreamChannel *stream = (StreamChannel*)streamID;
00560     \textcolor{keywordflow}{for}(\textcolor{keyword}{auto} i=mRxStreams.begin(); i!=mRxStreams.end(); ++i)
00561     \{
00562         \textcolor{keywordflow}{if}(*i==stream)
00563         \{
00564             \textcolor{keyword}{delete} *i;
00565             mRxStreams.erase(i);
00566             \textcolor{keywordflow}{break};
00567         \}
00568     \}
00569     \textcolor{keywordflow}{return} 0;
00570 \}
00571 
00572 \textcolor{keywordtype}{size\_t} ConnectionNovenaRF7::GetStreamSize(\textcolor{keyword}{const} \textcolor{keywordtype}{size\_t} streamID)
00573 \{
00574     \textcolor{keywordflow}{return} 16384;
00575 \}
00576 
00577 \textcolor{keywordtype}{int} ConnectionNovenaRF7::ControlStream(\textcolor{keyword}{const} \textcolor{keywordtype}{size\_t} streamID, \textcolor{keyword}{const} \textcolor{keywordtype}{bool} enable)
00578 \{
00579     \textcolor{keyword}{auto} *stream = (StreamChannel* )streamID;
00580     assert(stream != \textcolor{keyword}{nullptr});
00581 
00582     \textcolor{keywordflow}{if}(enable)
00583         \textcolor{keywordflow}{return} stream->Start();
00584     \textcolor{keywordflow}{else}
00585         \textcolor{keywordflow}{return} stream->Stop();
00586 \}
00587 
00588 \textcolor{keywordtype}{int} ConnectionNovenaRF7::ReadStreamStatus(\textcolor{keyword}{const} \textcolor{keywordtype}{size\_t} streamID, \textcolor{keyword}{const} \textcolor{keywordtype}{long} 
      timeout_ms, StreamMetadata &metadata)
00589 \{
00590     assert(streamID != 0);
00591     lime::IStreamChannel* channel = (lime::IStreamChannel*)streamID;
00592     StreamChannel::Info info = channel->GetInfo();
00593     metadata.hasTimestamp = \textcolor{keyword}{false};
00594     metadata.timestamp = info.timestamp;
00595     metadata.lateTimestamp = info.underrun > 0;
00596     metadata.packetDropped = info.droppedPackets > 0;
00597     \textcolor{keywordflow}{return} 0;
00598 \}
\end{DoxyCode}
