\subsection{L\+M\+S64\+C\+Protocol.\+cpp}
\label{LMS64CProtocol_8cpp_source}\index{/home/erik/prefix/default/src/limesuite-\/dev/src/protocols/\+L\+M\+S64\+C\+Protocol.\+cpp@{/home/erik/prefix/default/src/limesuite-\/dev/src/protocols/\+L\+M\+S64\+C\+Protocol.\+cpp}}

\begin{DoxyCode}
00001 
00007 \textcolor{preprocessor}{#include "Logger.h"}
00008 \textcolor{preprocessor}{#include "LMS64CProtocol.h"}
00009 \textcolor{preprocessor}{#include "Si5351C.h"}
00010 \textcolor{preprocessor}{#include <chrono>}
00011 \textcolor{preprocessor}{#include <iostream>}
00012 \textcolor{preprocessor}{#include <assert.h>}
00013 \textcolor{preprocessor}{#include <math.h>}
00014 \textcolor{preprocessor}{#include <algorithm>}
00015 \textcolor{preprocessor}{#include <iso646.h>} \textcolor{comment}{// alternative operators for visual c++: not, and, or...}
00016 \textcolor{preprocessor}{#include <ADCUnits.h>}
00017 \textcolor{preprocessor}{#include <sstream>}
00018 \textcolor{keyword}{using namespace }lime;
00019 
00021 \textcolor{keyword}{const} \textcolor{keywordtype}{int} LMS_RST_DEACTIVATE = 0;
00022 \textcolor{keyword}{const} \textcolor{keywordtype}{int} LMS_RST_ACTIVATE = 1;
00023 \textcolor{keyword}{const} \textcolor{keywordtype}{int} LMS_RST_PULSE = 2;
00024 
00026 \textcolor{preprocessor}{#define LMS7002M\_SPI\_INDEX 0x10}
00027 \textcolor{preprocessor}{#define ADF4002\_SPI\_INDEX 0x30}
00028 
00029 \textcolor{keyword}{static} \textcolor{keywordtype}{int} convertStatus(\textcolor{keyword}{const} \textcolor{keywordtype}{int} &status, \textcolor{keyword}{const} LMS64CProtocol::GenericPacket &pkt)
00030 \{
00031     \textcolor{keywordflow}{if} (status != 0) \textcolor{keywordflow}{return} -1;
00032     \textcolor{keywordflow}{switch} (pkt.status)
00033     \{
00034     \textcolor{keywordflow}{case} STATUS_COMPLETED_CMD: \textcolor{keywordflow}{return} 0;
00035     \textcolor{keywordflow}{case} STATUS_UNKNOWN_CMD:
00036         \textcolor{keywordflow}{return} ReportError(EPROTONOSUPPORT, \textcolor{stringliteral}{"Command not supported"});
00037     \textcolor{keywordflow}{default}: \textcolor{keywordflow}{break};
00038     \}
00039     \textcolor{keywordflow}{return} ReportError(EPROTO, status2string(pkt.status));
00040 \}
00041 
00042 LMS64CProtocol::LMS64CProtocol(\textcolor{keywordtype}{void})
00043 \{
00044     \textcolor{comment}{//set a sane-default for the rate}
00045     _cachedRefClockRate = 61.44e6/2;
00046 \}
00047 
00048 LMS64CProtocol::~LMS64CProtocol(\textcolor{keywordtype}{void})
00049 \{
00050     \textcolor{keywordflow}{return};
00051 \}
00052 
00053 \textcolor{keywordtype}{int} LMS64CProtocol::DeviceReset(\textcolor{keywordtype}{int} ind)
00054 \{
00055     \textcolor{keywordflow}{if} (not this->IsOpen())
00056     \{
00057         \textcolor{keywordflow}{return} ReportError(ENOTCONN, \textcolor{stringliteral}{"connection is not open"});
00058     \}
00059 
00060     GenericPacket pkt;
00061     pkt.cmd = CMD_LMS7002_RST;
00062     pkt.periphID = ind;
00063     pkt.outBuffer.push\_back (LMS_RST_PULSE);
00064     \textcolor{keywordtype}{int} status = this->TransferPacket(pkt);
00065 
00066     \textcolor{keywordflow}{return} convertStatus(status, pkt);
00067 \}
00068 
00069 \textcolor{keywordtype}{int} LMS64CProtocol::TransactSPI(\textcolor{keyword}{const} \textcolor{keywordtype}{int} addr, \textcolor{keyword}{const} uint32\_t *writeData, uint32\_t *readData, \textcolor{keyword}{const} \textcolor{keywordtype}{size\_t}
       size)
00070 \{
00074 
00075     \textcolor{keywordflow}{if} (not this->IsOpen())
00076     \{
00077         ReportError(ENOTCONN, \textcolor{stringliteral}{"connection is not open"});
00078         \textcolor{keywordflow}{return} -1;
00079     \}
00080 
00081     \textcolor{comment}{//perform spi writes when there is no read data}
00082     \textcolor{keywordflow}{if} (readData == \textcolor{keyword}{nullptr}) \textcolor{keywordflow}{switch}(addr)
00083     \{
00084     \textcolor{keywordflow}{case} LMS7002M_SPI_INDEX: \textcolor{keywordflow}{return} this->WriteLMS7002MSPI(writeData, size);
00085     \textcolor{keywordflow}{case} ADF4002_SPI_INDEX: \textcolor{keywordflow}{return} this->WriteADF4002SPI(writeData, size);
00086     \}
00087 
00088     \textcolor{comment}{//otherwise perform reads into the provided buffer}
00089     \textcolor{keywordflow}{if} (readData != \textcolor{keyword}{nullptr}) \textcolor{keywordflow}{switch}(addr)
00090     \{
00091     \textcolor{keywordflow}{case} LMS7002M_SPI_INDEX: \textcolor{keywordflow}{return} this->ReadLMS7002MSPI(writeData, readData, size);
00092     \textcolor{keywordflow}{case} ADF4002_SPI_INDEX: \textcolor{keywordflow}{return} this->ReadADF4002SPI(writeData, readData, size);
00093     \}
00094 
00095     \textcolor{keywordflow}{return} ReportError(ENOTSUP, \textcolor{stringliteral}{"unknown spi address"});
00096 \}
00097 
00098 \textcolor{keywordtype}{int} LMS64CProtocol::WriteI2C(\textcolor{keyword}{const} \textcolor{keywordtype}{int} addr, \textcolor{keyword}{const} std::string &data)
00099 \{
00100     \textcolor{keywordflow}{if} (not this->IsOpen())
00101     \{
00102         \textcolor{keywordflow}{return} ReportError(ENOTCONN, \textcolor{stringliteral}{"connection is not open"});
00103     \}
00104     \textcolor{keywordflow}{return} this->WriteSi5351I2C(data);
00105 \}
00106 
00107 \textcolor{keywordtype}{int} LMS64CProtocol::ReadI2C(\textcolor{keyword}{const} \textcolor{keywordtype}{int} addr, \textcolor{keyword}{const} \textcolor{keywordtype}{size\_t} numBytes, std::string &
      data)
00108 \{
00109     \textcolor{keywordflow}{if} (not this->IsOpen())
00110     \{
00111         \textcolor{keywordflow}{return} ReportError(ENOTCONN, \textcolor{stringliteral}{"connection is not open"});
00112     \}
00113     \textcolor{keywordflow}{return} this->ReadSi5351I2C(numBytes, data);
00114 \}
00115 
00116 \textcolor{comment}{/***********************************************************************}
00117 \textcolor{comment}{ * LMS7002M SPI access}
00118 \textcolor{comment}{ **********************************************************************/}
00119 \textcolor{keywordtype}{int} LMS64CProtocol::WriteLMS7002MSPI(\textcolor{keyword}{const} uint32\_t *writeData, \textcolor{keywordtype}{size\_t} size, \textcolor{keywordtype}{unsigned} periphID)
00120 \{
00121     GenericPacket pkt;
00122     pkt.cmd = CMD_LMS7002_WR;
00123     pkt.periphID = periphID;
00124     \textcolor{keywordflow}{for} (\textcolor{keywordtype}{size\_t} i = 0; i < size; ++i)
00125     \{
00126         uint16\_t addr = (writeData[i] >> 16) & 0x7fff;
00127         uint16\_t data = writeData[i] & 0xffff;
00128         pkt.outBuffer.push\_back(addr >> 8);
00129         pkt.outBuffer.push\_back(addr & 0xFF);
00130         pkt.outBuffer.push\_back(data >> 8);
00131         pkt.outBuffer.push\_back(data & 0xFF);
00132     \}
00133 
00134     \textcolor{keywordtype}{int} status = this->TransferPacket(pkt);
00135 
00136     \textcolor{keywordflow}{return} convertStatus(status, pkt);
00137 \}
00138 
00139 \textcolor{keywordtype}{int} LMS64CProtocol::ReadLMS7002MSPI(\textcolor{keyword}{const} uint32\_t *writeData, uint32\_t *readData, \textcolor{keywordtype}{size\_t} 
      size, \textcolor{keywordtype}{unsigned} periphID)
00140 \{
00141     GenericPacket pkt;
00142     pkt.cmd = CMD_LMS7002_RD;
00143     pkt.periphID = periphID;
00144     \textcolor{keywordflow}{for} (\textcolor{keywordtype}{size\_t} i = 0; i < size; ++i)
00145     \{
00146         uint16\_t addr = (writeData[i] >> 16) & 0x7fff;
00147         pkt.outBuffer.push\_back(addr >> 8);
00148         pkt.outBuffer.push\_back(addr & 0xFF);
00149     \}
00150 
00151     \textcolor{keywordtype}{int} status = this->TransferPacket(pkt);
00152 
00153     \textcolor{keyword}{const} \textcolor{keywordtype}{size\_t} numRead = std::min<size\_t>(pkt.inBuffer.size()/4, size);
00154     \textcolor{keywordflow}{for} (\textcolor{keywordtype}{size\_t} i = 0; i < numRead; ++i)
00155     \{
00156         \textcolor{keywordtype}{int} hi = pkt.inBuffer[4*i + 2];
00157         \textcolor{keywordtype}{int} lo = pkt.inBuffer[4*i + 3];
00158         readData[i] = (hi << 8) | lo;
00159     \}
00160 
00161     \textcolor{keywordflow}{return} convertStatus(status, pkt);
00162 \}
00163 
00164 \textcolor{comment}{/***********************************************************************}
00165 \textcolor{comment}{ * Si5351 SPI access}
00166 \textcolor{comment}{ **********************************************************************/}
00167 \textcolor{keywordtype}{int} LMS64CProtocol::WriteSi5351I2C(\textcolor{keyword}{const} std::string &data)
00168 \{
00169     GenericPacket pkt;
00170     pkt.cmd = CMD_SI5351_WR;
00171 
00172     \textcolor{keywordflow}{for} (\textcolor{keywordtype}{size\_t} i = 0; i < data.size(); i++)
00173     \{
00174         pkt.outBuffer.push\_back(data.at(i));
00175     \}
00176 
00177     \textcolor{keywordtype}{int} status = this->TransferPacket(pkt);
00178     \textcolor{keywordflow}{return} convertStatus(status, pkt);
00179 \}
00180 
00181 \textcolor{keywordtype}{int} LMS64CProtocol::ReadSi5351I2C(\textcolor{keyword}{const} \textcolor{keywordtype}{size\_t} numBytes, std::string &data)
00182 \{
00183     GenericPacket pkt;
00184     pkt.cmd = CMD_SI5351_RD;
00185 
00186     \textcolor{keywordtype}{int} status = this->TransferPacket(pkt);
00187 
00188     \textcolor{keywordflow}{for} (\textcolor{keywordtype}{size\_t} i = 0; i < data.size(); i++)
00189     \{
00190         pkt.outBuffer.push\_back(data.at(i));
00191     \}
00192 
00193     data.clear();
00194     \textcolor{keywordflow}{for} (\textcolor{keywordtype}{size\_t} i = 0; i < pkt.inBuffer.size(); ++i)
00195     \{
00196         data += pkt.inBuffer[i];
00197     \}
00198 
00199     \textcolor{keywordflow}{return} convertStatus(status, pkt);
00200 \}
00201 
00202 \textcolor{comment}{/***********************************************************************}
00203 \textcolor{comment}{ * ADF4002 SPI access}
00204 \textcolor{comment}{ **********************************************************************/}
00205 \textcolor{keywordtype}{int} LMS64CProtocol::WriteADF4002SPI(\textcolor{keyword}{const} uint32\_t *writeData, \textcolor{keyword}{const} \textcolor{keywordtype}{size\_t} size)
00206 \{
00207     GenericPacket pkt;
00208     pkt.cmd = CMD_ADF4002_WR;
00209 
00210     \textcolor{keywordflow}{for} (\textcolor{keywordtype}{size\_t} i = 0; i < size; i++)
00211     \{
00212         pkt.outBuffer.push\_back((writeData[i] >> 16) & 0xff);
00213         pkt.outBuffer.push\_back((writeData[i] >> 8) & 0xff);
00214         pkt.outBuffer.push\_back((writeData[i] >> 0) & 0xff);
00215     \}
00216 
00217     \textcolor{keywordtype}{int} status = this->TransferPacket(pkt);
00218     \textcolor{keywordflow}{return} convertStatus(status, pkt);
00219 \}
00220 
00221 \textcolor{keywordtype}{int} LMS64CProtocol::ReadADF4002SPI(\textcolor{keyword}{const} uint32\_t *writeData, uint32\_t *readData, \textcolor{keyword}{const} \textcolor{keywordtype}{size\_t} 
      size)
00222 \{
00223     ReportError(ENOTSUP, \textcolor{stringliteral}{"ReadADF4002SPI not supported"});
00224     \textcolor{keywordflow}{return} -1;
00225 \}
00226 
00227 \textcolor{comment}{/***********************************************************************}
00228 \textcolor{comment}{ * Board SPI access}
00229 \textcolor{comment}{ **********************************************************************/}
00230 \textcolor{keywordtype}{int} LMS64CProtocol::WriteRegisters(\textcolor{keyword}{const} uint32\_t *addrs, \textcolor{keyword}{const} uint32\_t *data, \textcolor{keyword}{const} \textcolor{keywordtype}{size\_t} 
      size)
00231 \{
00232     GenericPacket pkt;
00233     pkt.cmd = CMD_BRDSPI_WR;
00234     \textcolor{keywordflow}{for} (\textcolor{keywordtype}{size\_t} i = 0; i < size; ++i)
00235     \{
00236         pkt.outBuffer.push\_back(addrs[i] >> 8);
00237         pkt.outBuffer.push\_back(addrs[i] & 0xFF);
00238         pkt.outBuffer.push\_back(data[i] >> 8);
00239         pkt.outBuffer.push\_back(data[i] & 0xFF);
00240     \}
00241 
00242     \textcolor{keywordtype}{int} status = this->TransferPacket(pkt);
00243 
00244     \textcolor{keywordflow}{return} convertStatus(status, pkt);
00245 \}
00246 
00247 \textcolor{keywordtype}{int} LMS64CProtocol::ReadRegisters(\textcolor{keyword}{const} uint32\_t *addrs, uint32\_t *data, \textcolor{keyword}{const} \textcolor{keywordtype}{size\_t} 
      size)
00248 \{
00249     GenericPacket pkt;
00250     pkt.cmd = CMD_BRDSPI_RD;
00251     \textcolor{keywordflow}{for} (\textcolor{keywordtype}{size\_t} i = 0; i < size; ++i)
00252     \{
00253         pkt.outBuffer.push\_back(addrs[i] >> 8);
00254         pkt.outBuffer.push\_back(addrs[i] & 0xFF);
00255     \}
00256 
00257     \textcolor{keywordtype}{int} status = this->TransferPacket(pkt);
00258 
00259     \textcolor{keyword}{const} \textcolor{keywordtype}{size\_t} numRead = std::min<size\_t>(pkt.inBuffer.size()/4, size);
00260     \textcolor{keywordflow}{for} (\textcolor{keywordtype}{size\_t} i = 0; i < numRead; ++i)
00261     \{
00262         \textcolor{keywordtype}{int} hi = pkt.inBuffer[4*i + 2];
00263         \textcolor{keywordtype}{int} lo = pkt.inBuffer[4*i + 3];
00264         data[i] = (hi << 8) | lo;
00265     \}
00266 
00267     \textcolor{keywordflow}{return} convertStatus(status, pkt);
00268 \}
00269 
00270 \textcolor{comment}{/***********************************************************************}
00271 \textcolor{comment}{ * Device Information}
00272 \textcolor{comment}{ **********************************************************************/}
00273 DeviceInfo LMS64CProtocol::GetDeviceInfo(\textcolor{keywordtype}{void})
00274 \{
00275     LMSinfo lmsInfo = this->GetInfo();
00276     DeviceInfo devInfo;
00277     devInfo.deviceName = GetDeviceName(lmsInfo.device);
00278     devInfo.expansionName = GetExpansionBoardName(lmsInfo.expansion);
00279     devInfo.firmwareVersion = std::to\_string(\textcolor{keywordtype}{int}(lmsInfo.firmware));
00280     devInfo.hardwareVersion = std::to\_string(\textcolor{keywordtype}{int}(lmsInfo.hardware));
00281     devInfo.protocolVersion = std::to\_string(\textcolor{keywordtype}{int}(lmsInfo.protocol));
00282     devInfo.boardSerialNumber = lmsInfo.boardSerialNumber;
00283 
00284     FPGAinfo gatewareInfo = this->GetFPGAInfo();
00285     devInfo.gatewareTargetBoard = GetDeviceName(eLMS_DEV(gatewareInfo.boardID));
00286     devInfo.gatewareVersion = std::to\_string(\textcolor{keywordtype}{int}(gatewareInfo.gatewareVersion));
00287     devInfo.gatewareRevision = std::to\_string(\textcolor{keywordtype}{int}(gatewareInfo.gatewareRevision));
00288     devInfo.hardwareVersion = std::to\_string(\textcolor{keywordtype}{int}(gatewareInfo.hwVersion));
00289 
00290     \textcolor{keywordflow}{return} devInfo;
00291 \}
00292 
00295 LMS64CProtocol::LMSinfo LMS64CProtocol::GetInfo()
00296 \{
00297     LMSinfo info;
00298     info.device = LMS_DEV_UNKNOWN;
00299     info.expansion = EXP_BOARD_UNKNOWN;
00300     info.firmware = 0;
00301     info.hardware = 0;
00302     info.protocol = 0;
00303     info.boardSerialNumber = 0;
00304     GenericPacket pkt;
00305     pkt.cmd = CMD_GET_INFO;
00306     \textcolor{keywordtype}{int} status = TransferPacket(pkt);
00307     \textcolor{keywordflow}{if} (status == 0 && pkt.inBuffer.size() >= 5)
00308     \{
00309         info.firmware = pkt.inBuffer[0];
00310         info.device = pkt.inBuffer[1] < LMS_DEV_COUNT ? (eLMS_DEV)pkt.inBuffer[1] : 
      LMS_DEV_UNKNOWN;
00311         info.protocol = pkt.inBuffer[2];
00312         info.hardware = pkt.inBuffer[3];
00313         info.expansion = pkt.inBuffer[4] < EXP_BOARD_COUNT ? (eEXP_BOARD)pkt.
      inBuffer[4] : EXP_BOARD_UNKNOWN;
00314         info.boardSerialNumber = 0;
00315         \textcolor{keywordflow}{for} (\textcolor{keywordtype}{int} i = 10; i < 18; i++)
00316         \{
00317             info.boardSerialNumber <<= 8;
00318             info.boardSerialNumber |= pkt.inBuffer[i];
00319         \}
00320     \}
00321     \textcolor{keywordflow}{return} info;
00322 \}
00323 
00326 LMS64CProtocol::FPGAinfo LMS64CProtocol::GetFPGAInfo()
00327 \{
00328     FPGAinfo info;
00329     info.boardID = 0;
00330     info.gatewareVersion = 0;
00331     info.gatewareRevision = 0;
00332     GenericPacket pkt;
00333     pkt.cmd = CMD_BRDSPI_RD;
00334     \textcolor{keyword}{const} uint16\_t addrs[] = \{0x0000, 0x0001, 0x0002, 0x0003\};
00335     \textcolor{keywordflow}{for} (\textcolor{keywordtype}{size\_t} i = 0; i < 4; ++i)
00336     \{
00337         pkt.outBuffer.push\_back(addrs[i] >> 8);
00338         pkt.outBuffer.push\_back(addrs[i] & 0xFF);
00339     \}
00340     \textcolor{keywordtype}{int} status = this->TransferPacket(pkt);
00341     \textcolor{keywordflow}{if} (status == 0 && pkt.inBuffer.size() >= \textcolor{keyword}{sizeof}(addrs)*2)
00342     \{
00343         info.boardID = (pkt.inBuffer[2] << 8) | pkt.inBuffer[3];
00344         info.gatewareVersion = (pkt.inBuffer[6] << 8) | pkt.inBuffer[7];
00345         info.gatewareRevision = (pkt.inBuffer[10] << 8) | pkt.inBuffer[11];
00346         info.hwVersion = (pkt.inBuffer[14] << 8) | pkt.inBuffer[15];
00347     \}
00348     \textcolor{keywordflow}{return} info;
00349 \}
00350 
00351 
00356 \textcolor{keywordtype}{int} LMS64CProtocol::TransferPacket(GenericPacket& pkt)
00357 \{
00358     std::lock\_guard<std::mutex> lock(mControlPortLock);
00359     \textcolor{keywordtype}{int} status = 0;
00360     \textcolor{keywordflow}{if}(IsOpen() == \textcolor{keyword}{false}) ReportError(ENOTCONN, \textcolor{stringliteral}{"connection is not open"});
00361 
00362     \textcolor{keywordtype}{int} packetLen;
00363     eLMS_PROTOCOL protocol = LMS_PROTOCOL_UNDEFINED;
00364     \textcolor{keywordflow}{if}(this->GetType() == SPI_PORT)
00365         protocol = LMS_PROTOCOL_NOVENA;
00366     \textcolor{keywordflow}{else}
00367         protocol = LMS_PROTOCOL_LMS64C;
00368     \textcolor{keywordflow}{switch}(protocol)
00369     \{
00370     \textcolor{keywordflow}{case} LMS_PROTOCOL_UNDEFINED:
00371         \textcolor{keywordflow}{return} ReportError(\textcolor{stringliteral}{"protocol type undefined"});
00372     \textcolor{keywordflow}{case} LMS_PROTOCOL_LMS64C:
00373         packetLen = ProtocolLMS64C::pktLength;
00374         \textcolor{keywordflow}{break};
00375     \textcolor{keywordflow}{case} LMS_PROTOCOL_NOVENA:
00376         packetLen = pkt.outBuffer.size() > ProtocolNovena::pktLength ? 
      ProtocolNovena::pktLength : pkt.outBuffer.size();
00377         \textcolor{keywordflow}{break};
00378     \textcolor{keywordflow}{default}:
00379         packetLen = 0;
00380         \textcolor{keywordflow}{return} ReportError(\textcolor{stringliteral}{"Unknown protocol type %d"}, \textcolor{keywordtype}{int}(protocol));
00381     \}
00382     \textcolor{keywordtype}{int} outLen = 0;
00383     \textcolor{keywordtype}{unsigned} \textcolor{keywordtype}{char}* outBuffer = NULL;
00384     outBuffer = PreparePacket(pkt, outLen, protocol);
00385     \textcolor{keywordtype}{unsigned} \textcolor{keywordtype}{char}* inBuffer = \textcolor{keyword}{new} \textcolor{keywordtype}{unsigned} \textcolor{keywordtype}{char}[outLen];
00386     memset(inBuffer, 0, outLen);
00387 
00388     \textcolor{keywordtype}{int} outBufPos = 0;
00389     \textcolor{keywordtype}{int} inDataPos = 0;
00390     \textcolor{keywordflow}{if}(outLen == 0)
00391         outLen = 1;
00392 
00393     \textcolor{keywordflow}{if}(protocol == LMS_PROTOCOL_NOVENA)
00394     \{
00395         \textcolor{keywordtype}{bool} transferData = \textcolor{keyword}{true}; \textcolor{comment}{//some commands are fake, so don't need transferring}
00396         \textcolor{keywordflow}{if}(pkt.cmd == CMD_GET_INFO)
00397         \{
00398             \textcolor{comment}{//spi does not have GET INFO, fake it to inform what device it is}
00399             pkt.status = STATUS_COMPLETED_CMD;
00400             pkt.inBuffer.clear();
00401             pkt.inBuffer.resize(64, 0);
00402             pkt.inBuffer[0] = 0; \textcolor{comment}{//firmware}
00403             pkt.inBuffer[1] = LMS_DEV_NOVENA; \textcolor{comment}{//device}
00404             pkt.inBuffer[2] = 0; \textcolor{comment}{//protocol}
00405             pkt.inBuffer[3] = 0; \textcolor{comment}{//hardware}
00406             pkt.inBuffer[4] = EXP_BOARD_UNSUPPORTED; \textcolor{comment}{//expansion}
00407             transferData = \textcolor{keyword}{false};
00408         \}
00409 
00410         \textcolor{keywordflow}{if}(transferData)
00411         \{
00412             \textcolor{keywordflow}{if} (callback_logData)
00413                 callback_logData(\textcolor{keyword}{true}, outBuffer, outLen);
00414             \textcolor{keywordtype}{int} bytesWritten = Write(outBuffer, outLen);
00415             \textcolor{keywordflow}{if}( bytesWritten == outLen)
00416             \{
00417                 \textcolor{keywordflow}{if}(pkt.cmd == CMD_LMS7002_RD)
00418                 \{
00419                     inDataPos = Read(&inBuffer[inDataPos], outLen);
00420                     \textcolor{keywordflow}{if}(inDataPos != outLen)
00421                         status = ReportError(\textcolor{stringliteral}{"Read(%d bytes) got %d"}, (\textcolor{keywordtype}{int})outLen, (\textcolor{keywordtype}{int})inDataPos);
00422                     \textcolor{keywordflow}{else}
00423                     \{
00424                         \textcolor{keywordflow}{if} (callback_logData)
00425                             callback_logData(\textcolor{keyword}{false}, inBuffer, inDataPos);
00426                     \}
00427                 \}
00428                 ParsePacket(pkt, inBuffer, inDataPos, protocol);
00429             \}
00430             \textcolor{keywordflow}{else}
00431                 status = ReportError(\textcolor{stringliteral}{"Write(%d bytes) got %d"}, (\textcolor{keywordtype}{int})outLen, (\textcolor{keywordtype}{int})bytesWritten);
00432         \}
00433     \}
00434     \textcolor{keywordflow}{else}
00435     \{
00436         \textcolor{keywordflow}{for}(\textcolor{keywordtype}{int} i=0; i<outLen; i+=packetLen)
00437         \{
00438             \textcolor{keywordtype}{int} bytesToSend = packetLen;
00439             \textcolor{keywordflow}{if} (callback_logData)
00440                 callback_logData(\textcolor{keyword}{true}, &outBuffer[outBufPos], bytesToSend);
00441             \textcolor{keywordflow}{if}( Write(&outBuffer[outBufPos], bytesToSend) )
00442             \{
00443                 outBufPos += packetLen;
00444                 \textcolor{keywordtype}{long} readLen = packetLen;
00445                 \textcolor{keywordtype}{int} bread = Read(&inBuffer[inDataPos], readLen);
00446                 \textcolor{keywordflow}{if}(bread != readLen && protocol != LMS_PROTOCOL_NOVENA)
00447                 \{
00448                     status = ReportError(EIO, \textcolor{stringliteral}{"Read(%d bytes) failed"}, (\textcolor{keywordtype}{int})readLen);
00449                     \textcolor{keywordflow}{break};
00450                 \}
00451                 \textcolor{keywordflow}{if} (callback_logData)
00452                     callback_logData(\textcolor{keyword}{false}, &inBuffer[inDataPos], bread);
00453                 inDataPos += bread;
00454             \}
00455             \textcolor{keywordflow}{else}
00456             \{
00457                 status = ReportError(EIO, \textcolor{stringliteral}{"Write(%d bytes) failed"}, (\textcolor{keywordtype}{int})bytesToSend);
00458                 \textcolor{keywordflow}{break};
00459             \}
00460         \}
00461         ParsePacket(pkt, inBuffer, inDataPos, protocol);
00462     \}
00463     \textcolor{keyword}{delete}[] outBuffer;
00464     \textcolor{keyword}{delete}[] inBuffer;
00465     \textcolor{keywordflow}{return} convertStatus(status, pkt);
00466 \}
00467 
00474 \textcolor{keywordtype}{unsigned} \textcolor{keywordtype}{char}* LMS64CProtocol::PreparePacket(\textcolor{keyword}{const} GenericPacket& pkt, \textcolor{keywordtype}{int}& 
      length, \textcolor{keyword}{const} eLMS_PROTOCOL protocol)
00475 \{
00476     \textcolor{keywordtype}{unsigned} \textcolor{keywordtype}{char}* buffer = NULL;
00477     \textcolor{keywordflow}{if}(protocol == LMS_PROTOCOL_UNDEFINED)
00478         \textcolor{keywordflow}{return} NULL;
00479 
00480     \textcolor{keywordflow}{if}(protocol == LMS_PROTOCOL_LMS64C)
00481     \{
00482         ProtocolLMS64C packet;
00483         \textcolor{keywordtype}{int} maxDataLength = packet.maxDataLength;
00484         packet.cmd = pkt.cmd;
00485         packet.status = pkt.status;
00486         packet.periphID = pkt.periphID;
00487         \textcolor{keywordtype}{int} byteBlockRatio = 1; \textcolor{comment}{//block ratio - how many bytes in one block}
00488         \textcolor{keywordflow}{switch}( packet.cmd )
00489         \{
00490         \textcolor{keywordflow}{case} CMD_PROG_MCU:
00491         \textcolor{keywordflow}{case} CMD_GET_INFO:
00492         \textcolor{keywordflow}{case} CMD_SI5351_RD:
00493         \textcolor{keywordflow}{case} CMD_SI5356_RD:
00494             byteBlockRatio = 1;
00495             \textcolor{keywordflow}{break};
00496         \textcolor{keywordflow}{case} CMD_SI5351_WR:
00497         \textcolor{keywordflow}{case} CMD_SI5356_WR:
00498             byteBlockRatio = 2;
00499             \textcolor{keywordflow}{break};
00500         \textcolor{keywordflow}{case} CMD_LMS7002_RD:
00501         \textcolor{keywordflow}{case} CMD_BRDSPI_RD:
00502         \textcolor{keywordflow}{case} CMD_BRDSPI8_RD:
00503             byteBlockRatio = 2;
00504             \textcolor{keywordflow}{break};
00505         \textcolor{keywordflow}{case} CMD_ADF4002_WR:
00506             byteBlockRatio = 3;
00507             \textcolor{keywordflow}{break};
00508         \textcolor{keywordflow}{case} CMD_LMS7002_WR:
00509         \textcolor{keywordflow}{case} CMD_BRDSPI_WR:
00510         \textcolor{keywordflow}{case} CMD_ANALOG_VAL_WR:
00511             byteBlockRatio = 4;
00512             \textcolor{keywordflow}{break};
00513         \textcolor{keywordflow}{default}:
00514             byteBlockRatio = 1;
00515         \}
00516         \textcolor{keywordflow}{if} (packet.cmd == CMD_LMS7002_RD || packet.cmd == CMD_BRDSPI_RD)
00517             maxDataLength = maxDataLength/2;
00518         \textcolor{keywordflow}{if} (packet.cmd == CMD_ANALOG_VAL_RD)
00519             maxDataLength = maxDataLength / 4;
00520         \textcolor{keywordtype}{int} blockCount = pkt.outBuffer.size()/byteBlockRatio;
00521         \textcolor{keywordtype}{int} bufLen = blockCount/(maxDataLength/byteBlockRatio)
00522                     +(blockCount%(maxDataLength/byteBlockRatio)!=0);
00523         bufLen *= packet.pktLength;
00524         \textcolor{keywordflow}{if}(bufLen == 0)
00525             bufLen = packet.pktLength;
00526         buffer = \textcolor{keyword}{new} \textcolor{keywordtype}{unsigned} \textcolor{keywordtype}{char}[bufLen];
00527         memset(buffer, 0, bufLen);
00528         \textcolor{keywordtype}{unsigned} \textcolor{keywordtype}{int} srcPos = 0;
00529         \textcolor{keywordflow}{for}(\textcolor{keywordtype}{int} j=0; j*packet.pktLength<bufLen; ++j)
00530         \{
00531             \textcolor{keywordtype}{int} pktPos = j*packet.pktLength;
00532             buffer[pktPos] = packet.cmd;
00533             buffer[pktPos+1] = packet.status;
00534             buffer[pktPos + 3] = packet.periphID;
00535             \textcolor{keywordflow}{if}(blockCount > (maxDataLength/byteBlockRatio))
00536             \{
00537                 buffer[pktPos+2] = maxDataLength/byteBlockRatio;
00538                 blockCount -= buffer[pktPos+2];
00539             \}
00540             \textcolor{keywordflow}{else}
00541                 buffer[pktPos+2] = blockCount;
00542             memcpy(&buffer[pktPos+4], packet.reserved, \textcolor{keyword}{sizeof}(packet.reserved));
00543             \textcolor{keywordtype}{int} bytesToPack = (maxDataLength/byteBlockRatio)*byteBlockRatio;
00544             \textcolor{keywordflow}{for} (\textcolor{keywordtype}{int} k = 0; k<bytesToPack && srcPos < pkt.outBuffer.size(); ++srcPos, ++k)
00545                 buffer[pktPos + 8 + k] = pkt.outBuffer[srcPos];
00546         \}
00547         length = bufLen;
00548     \}
00549     \textcolor{keywordflow}{else} \textcolor{keywordflow}{if}(protocol == LMS_PROTOCOL_NOVENA)
00550     \{
00551         \textcolor{keywordflow}{if}(pkt.cmd == CMD_LMS7002_RST)
00552         \{
00553             buffer = \textcolor{keyword}{new} \textcolor{keywordtype}{unsigned} \textcolor{keywordtype}{char}[8];
00554             buffer[0] = 0x88;
00555             buffer[1] = 0x06;
00556             buffer[2] = 0x00;
00557             buffer[3] = 0x18;
00558             buffer[4] = 0x88;
00559             buffer[5] = 0x06;
00560             buffer[6] = 0x00;
00561             buffer[7] = 0x38;
00562             length = 8;
00563         \}
00564         \textcolor{keywordflow}{else}
00565         \{
00566             buffer = \textcolor{keyword}{new} \textcolor{keywordtype}{unsigned} \textcolor{keywordtype}{char}[pkt.outBuffer.size()];
00567             memcpy(buffer, &pkt.outBuffer[0], pkt.outBuffer.size());
00568             \textcolor{keywordflow}{if} (pkt.cmd == CMD_LMS7002_WR)
00569             \{
00570                 \textcolor{keywordflow}{for}(\textcolor{keywordtype}{size\_t} i=0; i<pkt.outBuffer.size(); i+=4)
00571                     buffer[i] |= 0x80;
00572             \}
00573             length = pkt.outBuffer.size();
00574         \}
00575     \}
00576     \textcolor{keywordflow}{return} buffer;
00577 \}
00578 
00586 \textcolor{keywordtype}{int} LMS64CProtocol::ParsePacket(GenericPacket& pkt, \textcolor{keyword}{const} \textcolor{keywordtype}{unsigned} \textcolor{keywordtype}{char}* buffer, \textcolor{keyword}{const} \textcolor{keywordtype}{int} 
      length, \textcolor{keyword}{const} eLMS_PROTOCOL protocol)
00587 \{
00588     \textcolor{keywordflow}{if}(protocol == LMS_PROTOCOL_UNDEFINED)
00589         \textcolor{keywordflow}{return} -1;
00590 
00591     \textcolor{keywordflow}{if}(protocol == LMS_PROTOCOL_LMS64C)
00592     \{
00593         ProtocolLMS64C packet;
00594         \textcolor{keywordtype}{int} inBufPos = 0;
00595         pkt.inBuffer.resize(packet.maxDataLength*(length / packet.pktLength + (length % packet.
      pktLength)), 0);
00596         \textcolor{keywordflow}{for}(\textcolor{keywordtype}{int} i=0; i<length; i+=packet.pktLength)
00597         \{
00598             pkt.cmd = (eCMD_LMS)buffer[i];
00599             pkt.status = (eCMD_STATUS)buffer[i+1];
00600             memcpy(&pkt.inBuffer[inBufPos], &buffer[i+8], packet.maxDataLength);
00601             inBufPos += packet.maxDataLength;
00602         \}
00603     \}
00604     \textcolor{keywordflow}{else} \textcolor{keywordflow}{if}(protocol == LMS_PROTOCOL_NOVENA)
00605     \{
00606         pkt.cmd = CMD_LMS7002_RD;
00607         pkt.status = STATUS_COMPLETED_CMD;
00608         pkt.inBuffer.clear();
00609         \textcolor{keywordflow}{for}(\textcolor{keywordtype}{int} i=0; i<length; i+=2)
00610         \{
00611             \textcolor{comment}{//reading from spi returns only registers values}
00612             \textcolor{comment}{//fill addresses as zeros to match generic format of address, value pairs}
00613             pkt.inBuffer.push\_back(0); \textcolor{comment}{//should be address msb}
00614             pkt.inBuffer.push\_back(0); \textcolor{comment}{//should be address lsb}
00615             pkt.inBuffer.push\_back(buffer[i]);
00616             pkt.inBuffer.push\_back(buffer[i+1]);
00617         \}
00618     \}
00619     \textcolor{keywordflow}{return} 1;
00620 \}
00621 
00622 \textcolor{keywordtype}{int} LMS64CProtocol::ProgramWrite(\textcolor{keyword}{const} \textcolor{keywordtype}{char} *data\_src, \textcolor{keyword}{const} \textcolor{keywordtype}{size\_t} length, \textcolor{keyword}{const} \textcolor{keywordtype}{int} prog\_mode, \textcolor{keyword}{const} \textcolor{keywordtype}{int} 
      device, ProgrammingCallback callback)
00623 \{
00624 \textcolor{preprocessor}{#ifndef NDEBUG}
00625     \textcolor{keyword}{auto} t1 = std::chrono::high\_resolution\_clock::now();
00626 \textcolor{preprocessor}{#endif}
00627     \textcolor{comment}{//erasing FLASH can take up to 3 seconds before reply is received}
00628     \textcolor{keyword}{const} \textcolor{keywordtype}{int} progTimeout\_ms = 5000;
00629     \textcolor{keywordtype}{char} progressMsg[128];
00630     sprintf(progressMsg, \textcolor{stringliteral}{"in progress..."});
00631     \textcolor{keywordtype}{bool} abortProgramming = \textcolor{keyword}{false};
00632     \textcolor{keywordtype}{int} bytesSent = 0;
00633 
00634     \textcolor{keywordtype}{bool} needsData = \textcolor{keyword}{true};
00635     \textcolor{keywordflow}{if}(device == FPGA && prog\_mode == 2)
00636         needsData = \textcolor{keyword}{false};
00637     \textcolor{keywordflow}{if}(device == FX3 && (prog\_mode == 0 || prog\_mode == 1))
00638         needsData = \textcolor{keyword}{false};
00639 
00640     \textcolor{keywordflow}{if}(length == 0 && needsData)
00641     \{
00642         \textcolor{keywordflow}{return} ReportError(EIO, \textcolor{stringliteral}{"ProgramWrite length should be > 0"});
00643     \}
00644 
00645     \textcolor{keywordflow}{if} (not this->IsOpen())
00646     \{
00647         \textcolor{keywordflow}{return} ReportError(ENOTCONN, \textcolor{stringliteral}{"connection is not open"});
00648     \}
00649 
00650     \textcolor{keyword}{const} \textcolor{keywordtype}{int} pktSize = 32;
00651     \textcolor{keywordtype}{int} data\_left = length;
00652     \textcolor{keyword}{const} \textcolor{keywordtype}{int} portionsCount = length/pktSize + (length%pktSize > 0) + 1; \textcolor{comment}{// +1 programming end packet}
00653     \textcolor{keywordtype}{int} portionNumber;
00654     \textcolor{keywordtype}{int} status = 0;
00655     eCMD_LMS cmd;
00656     \textcolor{keywordflow}{if}(device != FPGA)
00657         cmd = CMD_MEMORY_WR;
00658     \textcolor{keywordflow}{else} \textcolor{keywordflow}{if}(device == FPGA)
00659         cmd = CMD_ALTERA_FPGA_GW_WR;
00660     \textcolor{keywordflow}{else}
00661     \{
00662         sprintf(progressMsg, \textcolor{stringliteral}{"Programming failed! Target device not supported"});
00663         \textcolor{keywordflow}{if}(callback)
00664             abortProgramming = callback(bytesSent, length, progressMsg);
00665         \textcolor{keywordflow}{return} ReportError(ENOTSUP, progressMsg);
00666     \}
00667 
00668     \textcolor{keywordtype}{unsigned} \textcolor{keywordtype}{char} ctrbuf[64];
00669     \textcolor{keywordtype}{unsigned} \textcolor{keywordtype}{char} inbuf[64];
00670     memset(ctrbuf, 0, 64);
00671     ctrbuf[0] = cmd;
00672     ctrbuf[1] = 0;
00673     ctrbuf[2] = 56;
00674 
00675     \textcolor{keywordflow}{for} (portionNumber = 0; portionNumber<portionsCount && !abortProgramming; ++portionNumber)
00676     \{
00677         \textcolor{keywordtype}{int} offset = 8;
00678         memset(&ctrbuf[offset], 0, 56);
00679         ctrbuf[offset+0] = prog\_mode;
00680         ctrbuf[offset+1] = (portionNumber >> 24) & 0xFF;
00681         ctrbuf[offset+2] = (portionNumber >> 16) & 0xFF;
00682         ctrbuf[offset+3] = (portionNumber >> 8) & 0xFF;
00683         ctrbuf[offset+4] = portionNumber & 0xFF;
00684         \textcolor{keywordtype}{unsigned} \textcolor{keywordtype}{char} data\_cnt = data\_left > pktSize ? pktSize : data\_left;
00685         ctrbuf[offset+5] = data\_cnt;
00686         \textcolor{keywordflow}{if}(cmd == CMD_MEMORY_WR)
00687         \{
00688             ctrbuf[offset+6] = 0;
00689             ctrbuf[offset+7] = 0;
00690             ctrbuf[offset+8] = 0;
00691             ctrbuf[offset+9] = 0;
00692 
00693             ctrbuf[offset+10] = (device >> 8) & 0xFF;
00694             ctrbuf[offset+11] = device & 0xFF;
00695         \}
00696         \textcolor{keywordflow}{if}(data\_src != NULL)
00697         \{
00698             memcpy(&ctrbuf[offset + 24], data\_src, data\_cnt);
00699             data\_src += data\_cnt;
00700         \}
00701 
00702         \textcolor{keywordflow}{if}(Write(ctrbuf, \textcolor{keyword}{sizeof}(ctrbuf)) != \textcolor{keyword}{sizeof}(ctrbuf))
00703         \{
00704             \textcolor{keywordflow}{if}(callback)
00705                 callback(bytesSent, length, \textcolor{stringliteral}{"Programming failed! Write operation failed"});
00706             \textcolor{keywordflow}{return} ReportError(EIO, \textcolor{stringliteral}{"Programming failed! Write operation failed"});
00707         \}
00708         \textcolor{keywordflow}{if}(Read(inbuf, \textcolor{keyword}{sizeof}(inbuf), progTimeout\_ms) != \textcolor{keyword}{sizeof}(ctrbuf))
00709         \{
00710             \textcolor{keywordflow}{if}(callback)
00711                 callback(bytesSent, length, \textcolor{stringliteral}{"Programming failed! Read operation failed"});
00712             \textcolor{keywordflow}{return} ReportError(EIO, \textcolor{stringliteral}{"Programming failed! Read operation failed"});
00713         \}
00714         data\_left -= data\_cnt;
00715         status = inbuf[1];
00716         bytesSent += data\_cnt;
00717 
00718         \textcolor{keywordflow}{if}(status != STATUS_COMPLETED_CMD)
00719         \{
00720             sprintf(progressMsg, \textcolor{stringliteral}{"Programming failed! %s"}, status2string(status));
00721             \textcolor{keywordflow}{if}(callback)
00722                 abortProgramming = callback(bytesSent, length, progressMsg);
00723             \textcolor{keywordflow}{return} ReportError(EPROTO, progressMsg);
00724         \}
00725         \textcolor{keywordflow}{if}(needsData == \textcolor{keyword}{false}) \textcolor{comment}{//only one packet is needed to initiate bitstream from flash}
00726         \{
00727             bytesSent = length;
00728             \textcolor{keywordflow}{break};
00729         \}
00730         \textcolor{keywordflow}{if}(callback)
00731             abortProgramming = callback(bytesSent, length, progressMsg);
00732     \}
00733     \textcolor{keywordflow}{if} (abortProgramming == \textcolor{keyword}{true})
00734     \{
00735         sprintf(progressMsg, \textcolor{stringliteral}{"programming: aborted by user"});
00736         \textcolor{keywordflow}{if}(callback)
00737             callback(bytesSent, length, progressMsg);
00738         \textcolor{keywordflow}{return} ReportError(ECONNABORTED, \textcolor{stringliteral}{"user aborted programming"});
00739     \}
00740     sprintf(progressMsg, \textcolor{stringliteral}{"programming: completed"});
00741     \textcolor{keywordflow}{if}(callback)
00742         callback(bytesSent, length, progressMsg);
00743 \textcolor{preprocessor}{#ifndef NDEBUG}
00744     \textcolor{keyword}{auto} t2 = std::chrono::high\_resolution\_clock::now();
00745     \textcolor{keywordflow}{if} ((device == 2 && prog\_mode == 2) == \textcolor{keyword}{false})
00746         printf(\textcolor{stringliteral}{"Programming finished, %li bytes sent! %li ms\(\backslash\)n"}, length, 
      std::chrono::duration\_cast<std::chrono::milliseconds>(t2 - t1).count());
00747     \textcolor{keywordflow}{else}
00748         printf(\textcolor{stringliteral}{"FPGA configuring initiated\(\backslash\)n"});
00749 \textcolor{preprocessor}{#endif}
00750     \textcolor{keywordflow}{return} 0;
00751 \}
00752 
00753 \textcolor{keywordtype}{int} LMS64CProtocol::CustomParameterRead(\textcolor{keyword}{const} uint8\_t *ids, \textcolor{keywordtype}{double} *values, \textcolor{keyword}{const} \textcolor{keywordtype}{size\_t} 
      count, std::string* units)
00754 \{
00755     LMS64CProtocol::GenericPacket pkt;
00756     pkt.cmd = CMD_ANALOG_VAL_RD;
00757 
00758     \textcolor{keywordflow}{for} (\textcolor{keywordtype}{size\_t} i=0; i<count; ++i)
00759         pkt.outBuffer.push\_back(ids[i]);
00760 
00761     \textcolor{keywordtype}{int} status = this->TransferPacket(pkt);
00762     \textcolor{keywordflow}{if} (status != 0) \textcolor{keywordflow}{return} status;
00763 
00764     assert(pkt.inBuffer.size() >= 4 * count);
00765 
00766     \textcolor{keywordflow}{for} (\textcolor{keywordtype}{size\_t} i = 0; i < count; ++i)
00767     \{
00768         \textcolor{keywordtype}{int} unitsIndex = pkt.inBuffer[i * 4 + 1];
00769         \textcolor{keywordflow}{if}(units)
00770         \{
00771 
00772             \textcolor{keyword}{const} \textcolor{keywordtype}{char} adc\_units\_prefix[] = \{
00773                 \textcolor{charliteral}{' '}, \textcolor{charliteral}{'k'}, \textcolor{charliteral}{'M'}, \textcolor{charliteral}{'G'}, \textcolor{charliteral}{'T'}, \textcolor{charliteral}{'P'}, \textcolor{charliteral}{'E'}, \textcolor{charliteral}{'Z'},
00774                 \textcolor{charliteral}{'y'}, \textcolor{charliteral}{'z'}, \textcolor{charliteral}{'a'}, \textcolor{charliteral}{'f'}, \textcolor{charliteral}{'p'}, \textcolor{charliteral}{'n'}, \textcolor{charliteral}{'u'}, \textcolor{charliteral}{'m'}\};
00775             units[i] = adc\_units\_prefix[unitsIndex&0x0F];
00776             units[i] += adcUnits2string((unitsIndex & 0xF0)>>4);
00777         \}
00778         values[i] = (int16\_t)(pkt.inBuffer[i * 4 + 2] << 8 | pkt.inBuffer[i * 4 + 3]);
00779 
00780         \textcolor{keywordflow}{if}((unitsIndex & 0xF0)>>4 == TEMPERATURE)
00781             values[i] /= 10;
00782     \}
00783     \textcolor{keywordflow}{return} 0;
00784 \}
00785 
00786 \textcolor{keywordtype}{int} LMS64CProtocol::CustomParameterWrite(\textcolor{keyword}{const} uint8\_t *ids, \textcolor{keyword}{const} \textcolor{keywordtype}{double} *values, \textcolor{keyword}{const} \textcolor{keywordtype}{size\_t} 
      count, \textcolor{keyword}{const} std::string& units)
00787 \{
00788     LMS64CProtocol::GenericPacket pkt;
00789     pkt.cmd = CMD_ANALOG_VAL_WR;
00790 
00791     \textcolor{keywordflow}{for} (\textcolor{keywordtype}{size\_t} i = 0; i < count; ++i)
00792     \{
00793         pkt.outBuffer.push\_back(ids[i]);
00794         \textcolor{keywordtype}{int} powerOf10 = 0;
00795         \textcolor{keywordflow}{if}(values[i] > 65535.0 && (units != \textcolor{stringliteral}{""}))
00796             powerOf10 = log10(values[i]/65.536)/3;
00797         \textcolor{keywordflow}{if} (values[i] < 65.536 && (units != \textcolor{stringliteral}{""}))
00798             powerOf10 = log10(values[i]/65535.0) / 3;
00799         \textcolor{keywordtype}{int} unitsId = 0; \textcolor{comment}{// need to convert given units to their enum}
00800         pkt.outBuffer.push\_back(unitsId << 4 | powerOf10);
00801         \textcolor{keywordtype}{int} value = values[i] / pow(10, 3*powerOf10);
00802         pkt.outBuffer.push\_back(value >> 8);
00803         pkt.outBuffer.push\_back(value & 0xFF);
00804     \}
00805     \textcolor{keywordflow}{return} TransferPacket(pkt);
00806 \}
00807 
00808 \textcolor{keywordtype}{int} LMS64CProtocol::GPIOWrite(\textcolor{keyword}{const} uint8\_t *buffer, \textcolor{keyword}{const} \textcolor{keywordtype}{size\_t} bufLength)
00809 \{
00810     LMS64CProtocol::GenericPacket pkt;
00811     pkt.cmd = CMD_GPIO_WR;
00812     \textcolor{keywordflow}{for} (\textcolor{keywordtype}{size\_t} i=0; i<bufLength; ++i)
00813         pkt.outBuffer.push\_back(buffer[i]);
00814     \textcolor{keywordflow}{return} TransferPacket(pkt);
00815 \}
00816 
00817 \textcolor{keywordtype}{int} LMS64CProtocol::GPIORead(uint8\_t *buffer, \textcolor{keyword}{const} \textcolor{keywordtype}{size\_t} bufLength)
00818 \{
00819     LMS64CProtocol::GenericPacket pkt;
00820     pkt.cmd = CMD_GPIO_RD;
00821     \textcolor{keywordtype}{int} status = TransferPacket(pkt);
00822     \textcolor{keywordflow}{if}(status != 0)
00823         \textcolor{keywordflow}{return} status;
00824 
00825     \textcolor{keywordflow}{for} (\textcolor{keywordtype}{size\_t} i=0; i<bufLength; ++i)
00826         buffer[i] = pkt.inBuffer[i];
00827     \textcolor{keywordflow}{return} status;
00828 \}
00829 
00830 \textcolor{keywordtype}{int} LMS64CProtocol::GPIODirWrite(\textcolor{keyword}{const} uint8\_t *buffer, \textcolor{keyword}{const} \textcolor{keywordtype}{size\_t} bufLength)
00831 \{
00832     LMS64CProtocol::GenericPacket pkt;
00833     pkt.cmd = CMD_GPIO_DIR_WR;
00834     \textcolor{keywordflow}{for} (\textcolor{keywordtype}{size\_t} i=0; i<bufLength; ++i)
00835         pkt.outBuffer.push\_back(buffer[i]);
00836     \textcolor{keywordflow}{return} TransferPacket(pkt);
00837 \}
00838 
00839 \textcolor{keywordtype}{int} LMS64CProtocol::GPIODirRead(uint8\_t *buffer, \textcolor{keyword}{const} \textcolor{keywordtype}{size\_t} bufLength)
00840 \{
00841     LMS64CProtocol::GenericPacket pkt;
00842     pkt.cmd = CMD_GPIO_DIR_RD;
00843     \textcolor{keywordtype}{int} status = TransferPacket(pkt);
00844     \textcolor{keywordflow}{if}(status != 0)
00845         \textcolor{keywordflow}{return} convertStatus(status, pkt);
00846 
00847     \textcolor{keywordflow}{for} (\textcolor{keywordtype}{size\_t} i=0; i<bufLength; ++i)
00848         buffer[i] = pkt.inBuffer[i];
00849     \textcolor{keywordflow}{return} status;
00850 \}
00851 
00852 \textcolor{keywordtype}{int} LMS64CProtocol::ProgramMCU(\textcolor{keyword}{const} uint8\_t *buffer, \textcolor{keyword}{const} \textcolor{keywordtype}{size\_t} length, \textcolor{keyword}{const} 
      MCU_PROG_MODE mode, ProgrammingCallback callback)
00853 \{
00854 \textcolor{preprocessor}{#ifndef NDEBUG}
00855     \textcolor{keyword}{auto} timeStart = std::chrono::high\_resolution\_clock::now();
00856 \textcolor{preprocessor}{#endif}
00857     \textcolor{keyword}{const} uint8\_t fifoLen = 32;
00858     \textcolor{keywordtype}{bool} success = \textcolor{keyword}{true};
00859     \textcolor{keywordtype}{bool} terminate = \textcolor{keyword}{false};
00860 
00861     \textcolor{keywordtype}{int} packetNumber = 0;
00862     \textcolor{keywordtype}{int} status = STATUS_UNDEFINED;
00863 
00864     LMS64CProtocol::GenericPacket pkt;
00865     pkt.cmd = CMD_PROG_MCU;
00866 
00867     \textcolor{keywordflow}{if} (callback)
00868         terminate = callback(0, length,\textcolor{stringliteral}{""});
00869 
00870     \textcolor{keywordflow}{for}(uint16\_t CntEnd=0; CntEnd<length && !terminate; CntEnd+=32)
00871     \{
00872         pkt.outBuffer.clear();
00873         pkt.outBuffer.reserve(fifoLen+2);
00874         pkt.outBuffer.push\_back(mode);
00875         pkt.outBuffer.push\_back(packetNumber++);
00876         \textcolor{keywordflow}{for} (uint8\_t i=0; i<fifoLen; i++)
00877             pkt.outBuffer.push\_back(buffer[CntEnd + i]);
00878 
00879         TransferPacket(pkt);
00880         status = pkt.status;
00881         \textcolor{keywordflow}{if} (callback)
00882             terminate = callback(CntEnd+fifoLen,length,\textcolor{stringliteral}{""});
00883 \textcolor{preprocessor}{#ifndef NDEBUG}
00884         printf(\textcolor{stringliteral}{"MCU programming : %4i/%4li\(\backslash\)r"}, CntEnd+fifoLen, \textcolor{keywordtype}{long}(length));
00885 \textcolor{preprocessor}{#endif}
00886         \textcolor{keywordflow}{if}(status != STATUS_COMPLETED_CMD)
00887         \{
00888             std::stringstream ss;
00889             ss << \textcolor{stringliteral}{"Programming MCU: status : not completed, block "} << packetNumber << std::endl;
00890             success = \textcolor{keyword}{false};
00891             \textcolor{keywordflow}{break};
00892         \}
00893 
00894         \textcolor{keywordflow}{if}(mode == 3) \textcolor{comment}{// if boot mode , send only first packet}
00895         \{
00896             \textcolor{keywordflow}{if} (callback)
00897                 callback(1, 1, \textcolor{stringliteral}{""});
00898             \textcolor{keywordflow}{break};
00899         \}
00900     \};
00901 \textcolor{preprocessor}{#ifndef NDEBUG}
00902     \textcolor{keyword}{auto} timeEnd = std::chrono::high\_resolution\_clock::now();
00903     printf(\textcolor{stringliteral}{"\(\backslash\)nMCU Programming finished, %li ms\(\backslash\)n"},
00904             std::chrono::duration\_cast<std::chrono::milliseconds>
00905             (timeEnd-timeStart).count());
00906 \textcolor{preprocessor}{#endif}
00907     \textcolor{keywordflow}{return} success ? 0 : -1;
00908 \}
00909 
\end{DoxyCode}
