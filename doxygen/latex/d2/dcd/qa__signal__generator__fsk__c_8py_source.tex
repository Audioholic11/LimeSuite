\subsection{qa\+\_\+signal\+\_\+generator\+\_\+fsk\+\_\+c.\+py}
\label{qa__signal__generator__fsk__c_8py_source}\index{/home/erik/prefix/default/src/gr-\/radar-\/dev/python/qa\+\_\+signal\+\_\+generator\+\_\+fsk\+\_\+c.\+py@{/home/erik/prefix/default/src/gr-\/radar-\/dev/python/qa\+\_\+signal\+\_\+generator\+\_\+fsk\+\_\+c.\+py}}

\begin{DoxyCode}
00001 \textcolor{comment}{#!/usr/bin/env python}
00002 \textcolor{comment}{# -*- coding: utf-8 -*-}
00003 \textcolor{comment}{# }
00004 \textcolor{comment}{# Copyright 2014 Communications Engineering Lab, KIT.}
00005 \textcolor{comment}{# }
00006 \textcolor{comment}{# This is free software; you can redistribute it and/or modify}
00007 \textcolor{comment}{# it under the terms of the GNU General Public License as published by}
00008 \textcolor{comment}{# the Free Software Foundation; either version 3, or (at your option)}
00009 \textcolor{comment}{# any later version.}
00010 \textcolor{comment}{# }
00011 \textcolor{comment}{# This software is distributed in the hope that it will be useful,}
00012 \textcolor{comment}{# but WITHOUT ANY WARRANTY; without even the implied warranty of}
00013 \textcolor{comment}{# MERCHANTABILITY or FITNESS FOR A PARTICULAR PURPOSE.  See the}
00014 \textcolor{comment}{# GNU General Public License for more details.}
00015 \textcolor{comment}{# }
00016 \textcolor{comment}{# You should have received a copy of the GNU General Public License}
00017 \textcolor{comment}{# along with this software; see the file COPYING.  If not, write to}
00018 \textcolor{comment}{# the Free Software Foundation, Inc., 51 Franklin Street,}
00019 \textcolor{comment}{# Boston, MA 02110-1301, USA.}
00020 \textcolor{comment}{# }
00021 
00022 \textcolor{keyword}{from} gnuradio \textcolor{keyword}{import} gr, gr\_unittest
00023 \textcolor{keyword}{from} gnuradio \textcolor{keyword}{import} blocks
00024 \textcolor{keyword}{import} radar\_swig \textcolor{keyword}{as} radar
00025 \textcolor{keyword}{import} numpy \textcolor{keyword}{as} np
00026 \textcolor{keyword}{import} numpy.fft
00027 
00028 \textcolor{keyword}{class }qa_signal_generator_fsk_c (gr\_unittest.TestCase):
00029 
00030     \textcolor{keyword}{def }setUp (self):
00031         self.tb = gr.top\_block ()
00032 
00033     \textcolor{keyword}{def }tearDown (self):
00034         self.tb = \textcolor{keywordtype}{None}
00035 
00036     \textcolor{keyword}{def }test_001_t (self):
00037         \textcolor{comment}{# compare signal in time domain}
00038         \textcolor{comment}{# set up fg}
00039         test\_len = 1000
00040         
00041         samp\_rate = 4000
00042         samp\_per\_freq = 2
00043         blocks\_per\_tag = 100
00044         freq\_low = 0
00045         freq\_high = 200
00046         amplitude = 1
00047         
00048         src = radar.signal\_generator\_fsk\_c(samp\_rate, samp\_per\_freq, blocks\_per\_tag, freq\_low, freq\_high, 
      amplitude)
00049         head = blocks.head(8,test\_len)
00050         snk = blocks.vector\_sink\_c()
00051         
00052         self.tb.connect(src,head,snk)
00053         self.tb.run ()
00054         
00055         \textcolor{comment}{# get ref data}
00056         ref\_data = [0]*test\_len
00057         phase\_low = 0
00058         phase\_high = 0
00059         counter = 0
00060         state = 0
00061         \textcolor{keywordflow}{for} k \textcolor{keywordflow}{in} range(len(ref\_data)):
00062             \textcolor{keywordflow}{if} counter==samp\_per\_freq:
00063                 \textcolor{keywordflow}{if} state:
00064                     state = 0
00065                 \textcolor{keywordflow}{else}:
00066                     state = 1
00067                 counter = 0
00068             
00069             \textcolor{keywordflow}{if} state:
00070                 ref\_data[k] = amplitude*np.exp(1j*phase\_high)
00071             \textcolor{keywordflow}{else}:
00072                 ref\_data[k] = amplitude*np.exp(1j*phase\_low) \textcolor{comment}{# first is low}
00073                 
00074             phase\_low = phase\_low+2*np.pi*freq\_low/samp\_rate
00075             phase\_high = phase\_high+2*np.pi*freq\_high/samp\_rate
00076             counter = counter+1
00077         
00078         \textcolor{comment}{# check data}
00079         data = snk.data()
00080         self.assertComplexTuplesAlmostEqual(data,ref\_data,4)
00081         
00082     \textcolor{keyword}{def }test_002_t (self):
00083         \textcolor{comment}{# compare signal in frequency domain}
00084         \textcolor{comment}{# set up fg}
00085         test\_len = 1000
00086         
00087         samp\_rate = 4000
00088         samp\_per\_freq = 1
00089         blocks\_per\_tag = 100
00090         freq\_low = 0
00091         freq\_high = 200
00092         amplitude = 1
00093         
00094         src = radar.signal\_generator\_fsk\_c(samp\_rate, samp\_per\_freq, blocks\_per\_tag, freq\_low, freq\_high, 
      amplitude)
00095         head = blocks.head(8,test\_len)
00096         snk = blocks.vector\_sink\_c()
00097         
00098         self.tb.connect(src,head,snk)
00099         self.tb.run ()
00100         
00101         \textcolor{comment}{# split data}
00102         data = snk.data()
00103         data0 = [0]*(test\_len/2)
00104         data1 = [0]*(test\_len/2)
00105         
00106         \textcolor{keywordflow}{for} k \textcolor{keywordflow}{in} range(test\_len/2):
00107             data0[k] = data[2*k]
00108             data1[k] = data[2*k+1]
00109         
00110         \textcolor{comment}{# check data with fft}
00111         fft0 = numpy.fft.fft(data0) \textcolor{comment}{# get fft}
00112         fft1 = numpy.fft.fft(data1) \textcolor{comment}{# get fft}
00113         num0 = np.argmax(abs(fft0)) \textcolor{comment}{# index of max sample}
00114         fft\_freq0 = samp\_rate/2*num0/len(fft0) \textcolor{comment}{# calc freq out of max sample index, works only for
       frequencies < samp\_rate/2!, samp rate is halfed}
00115         num1 = np.argmax(abs(fft1)) \textcolor{comment}{# index of max sample}
00116         fft\_freq1 = samp\_rate/2*num1/len(fft1) \textcolor{comment}{# calc freq out of max sample index, works only for
       frequencies < samp\_rate/2!, samp rate is halfed}
00117         
00118         self.assertEqual(freq\_low,fft\_freq0) \textcolor{comment}{# check if freq is correct}
00119         self.assertEqual(freq\_high,fft\_freq1)
00120 
00121 \textcolor{keywordflow}{if} \_\_name\_\_ == \textcolor{stringliteral}{'\_\_main\_\_'}:
00122     gr\_unittest.run(qa\_signal\_generator\_fsk\_c)\textcolor{comment}{#, "qa\_signal\_generator\_fsk\_c.xml")}
\end{DoxyCode}
